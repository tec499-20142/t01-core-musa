%
% Portuguese-BR vertion
% 
\documentclass{article}

\usepackage{ipprocess}
% Use longtable if you want big tables to split over multiple pages.
% \usepackage{longtable}
\usepackage[utf8]{inputenc} 
% Babel package is used to translate keywors to a specific language. 
% The option "brazil" defines portuguese-brazil as default language.
% Babel is also useful for hiphenation.
\usepackage[brazil]{babel}

\sloppy

\graphicspath{{./pictures/}} % Pictures dir
\makeindex
\begin{document}

\DocumentTitle{Especificação de Requisitos}
\Project{Processador de Propósito Geral Simples}
\Organization{Universidade Estadual de Feira de Santana}
\Version{Compilação 1.2}
\capa

\newpage

%%%%%%%%%%%%%%%%%%%%%%%%%%%%%%%%%%%%%%%%%%%%%%%%%%
%% Revision History
%%%%%%%%%%%%%%%%%%%%%%%%%%%%%%%%%%%%%%%%%%%%%%%%%%
\section*{\center Histórico de Revisões}
  \vspace*{1cm}
  \begin{table}[ht]
    \centering
    \begin{tabular}[pos]{|m{2cm} | m{7.2cm} | m{3.8cm}|} 
      \hline
      \cellcolor[gray]{0.9}
      \textbf{Data} & \cellcolor[gray]{0.9}\textbf{Descrição} & \cellcolor[gray]{0.9}\textbf{Autor(es)}\\ \hline
      \hline
      \small 26/09/2014 & \small Inicialização do documento & \small patrickecomp \\ \hline      
      \small 26/09/2014 &
      \small Inclusão dos requisitos [FR1] à [FR8] & \small patrickecomp \\ \hline 
      \small 28/09/2014 & \small Inclusão dos requisitos [FR11] à [FR15] & \small santana22 e gabri14el \\ \hline
      \small 28/09/2014 & \small Concepção da introdução e inclusão dos requisitos [FR9] e [FR10] & \small manuellemacedo \\ \hline 
            \small 29/09/2014 & \small Inclusão dos requisitos [FR17]  [FR22] & \small patrickecomp, mirelarios e tarles-w7 \\ \hline   
                        \small 29/09/2014 & \small Inclusão dos requisitos [FR23] e [FR24] & \small ardersonqdv, lucas \\ \hline 
                                                \small 29/09/2014 & \small Inclusão do requisito [FR7] & \small patrickecomp \\ \hline 
    \end{tabular}
  \end{table}

\newpage

% TOC instantiation
\tableofcontents
\newpage

%%%%%%%%%%%%%%%%%%%%%%%%%%%%%%%%%%%%%%%%%%%%%%%%%%
%% Document main content
%%%%%%%%%%%%%%%%%%%%%%%%%%%%%%%%%%%%%%%%%%%%%%%%%%
\section{Introdução}

 A \textbf{Fazemos Qualquer Negócio Inc.} foi contratada para o desenvolvimento do \textit{IP-Core} \textbf{MUSA}, que é um micro processador de
 propósito geral que será utilizado em escolas de países africanos com o intuito de impulsionar o desenvolvimento deste continente. As seções
 subsequentes apresentam os requisitos funcionais e não funcionais como parte da fase de \textbf{Concepção} do IP-process.
 Os requisitos listados foram definidos a partir das necessidades do cliente para produção do \textit{MUSA}.    

\subsection{Visão Geral do Documento}
  \begin{itemize}
   \item \textbf{Requisitos funcionais -} lista de todos os requisitos funcionais.
   \item \textbf{Requisitos não funcionais -} lista de todos os requisitos não funcionais.
   \item \textbf{Dependências -} conjunto de dependências de IP-cores previstos.
   % \item \textbf{Notas -} apresenta a lista de notas apresentadas ao longo do documento.
   % \item \textbf{Referências -} lista de todos os textos referenciados nesse documento.
  \end{itemize}

  \subsection{Definições}
  
  \FloatBarrier
  \begin{table}[H]
    \begin{center}
      \begin{tabular}[pos]{|m{2.5cm} | m{11.5cm}|} 
        \hline
        \cellcolor[gray]{0.9}\textbf{Termo} & \cellcolor[gray]{0.9}\textbf{Descrição} \\ \hline
        \flushright \centering {Requisito Funcional} & Requisitos de hardware que compõem os módulos, descrevendo as ações que o 
                                    mesmo deve estar apto a executar. Estas informações são capturadas a partir 
                                    do desenvolvimento dos casos de uso, que documentam as entradas, os processos 
                                    e as saídas geradas.  \\ \hline
        \flushright \centering {Requisito Não \mbox{Funcional}} & Requisitos de hardware que compõem os módulos, representando as características 
                                    que o mesmo deve ter, ou restrições que o mesmo deve operar. Estas características
                                    referem-se a técnicas, algoritmos, tecnologias e especificidades do Sistema como um todo.  \\ \hline
        \flushright \centering {Dependências}              & Requisitos de reuso de IP-cores, descrevendo as funções que cada um deve exercer. \\ \hline
      \end{tabular}
    \end{center}
  \end{table}  

  % inicio da tabela de acronimos e abreviacoes do documento
  \subsection{Acrônimos e Abreviações}
    \FloatBarrier
    \begin{table}[H]
      \begin{center}
        \begin{tabular}[pos]{|m{2cm} | m{12cm}|} 
          \hline
          \cellcolor[gray]{0.9}\textbf{Sigla} & \cellcolor[gray]{0.9}\textbf{Descrição} \\ \hline
          FR      & Requisito Funcional  \\ \hline
          NFR     & Requisito Não Funcional  \\ \hline
          D       & Dependência  \\ \hline
		  PC & Program Counter \\ \hline 
        \end{tabular}
      \end{center}
    \end{table}  
  % fim  

  % inicio da descriao de prioridades de requisitos
  \subsection{Prioridades dos Requisitos}
    \FloatBarrier
    \begin{table}[H]
      \begin{center}
        \begin{tabular}[pos]{|m{2cm} | m{12cm}|} 
          \hline
          \cellcolor[gray]{0.9}\textbf{Prioridade} & \cellcolor[gray]{0.9}\textbf{Característica} \\ \hline
          Importante      & Requisito sem o qual o sistema funciona, porém não como deveria.  \\ \hline
          Essencial       & Requisito que deve ser implementado para que o sistema funcione.  \\ \hline
          Desejável       & Requisito que não compromete o funcionamento do sistema.  \\ \hline
        \end{tabular}
      \end{center}
    \end{table}  
  % fim

\section{Requisitos Funcionais}


  \subsection{Conjunto de Operações Aritméticas} 
  
    \begin{functional}
      \requirement
      {Operação de soma}
      {O módulo deve ser capaz de realizar a operação de soma de dois valores de 8 bits.}
      {Essencial}

      \requirement
      {Operação de subtração}
      {O módulo deve ser capaz de realizar a operação de subtração de dois valores de 8 bits.}
      {Essencial}

      \requirement
      {Operação de multiplicação}
      {O módulo deve ser capaz de realizar a operação de multiplicação de dois valores de 8 bits. }
      {Essencial}

      \requirement
      {Operação de divisão}
      {O módulo deve ser capaz de realizar a operação de divisão de dois valores de 8 bits.}
      {Essencial} 

      \requirement
      {Tamanho da palavra de saída}
      {O componente deve apresentar uma saída única de 8 bits para todas as operações aritméticas.}
      {Importante}       

      \requirement
      {Detecção de overflow aritmético}
      {O módulo deve ser capaz de detectar \textit{overflow} aritmético.}
      {Importante}
      
      \requirement
      {Detecção de zero}
      {O módulo deve ser capaz de detectar no resultado da operação o valor zero e habilitar uma flag.}
      [Importante]
    \end{functional}

  \subsection{Conjunto de Operações Lógicas} 
  
    \begin{functional}
      \requirement{Operação AND}
      {O módulo deve ser capaz de realizar a operação AND lógico de dois operandos de 8 bits.}
      {Essencial}

      \requirement{Operação OR}
      {O módulo deve ser capaz de realizar a operação OR lógico de dois operandos de 8 bits.}
      {Essencial}  
      
      \requirement{Operação NOT}
      {O módulo deve ser capaz de realizar a operação NOT lógico de um operando de 8 bits.}
      {Essencial}   
      
      \requirement{Operação CMP}
      {O módulo deve ser capaz de comparar dois registradores de 8 bits cada e ativar um flag}
      {Essencial}
    \end{functional}  
    
    \subsection{Conjunto de Operações de Desvio}
	    
	  \begin{functional}
		\requirement{Operação JR}
		{O módulo deve ser capaz de realizar desvios para endereços específicos na memória.}
		{Essencial} 
		 
		 \requirement{Operação JPC}
		 {O módulo deve ser capaz de realizar desvios para endereços específicos na memória relativos ao PC.}
		 {Essencial}
		 
		 \requirement{Operação BRFL}
		 {O módulo deve ser capaz de realizar desvios condicionais com base em um \textit{flag}.}
		 {Essencial}
		 
		 \requirement{Operação CALL}
		 {O módulo deve ser capaz de desviar a execução para uma sub-rotina.}
		 {Essencial}
		 
		 \requirement{Operação RET}
		 {O módulo deve ser capaz de retornar a execução de uma sub-rotina.}
		 {Essencial}
         
	  \end{functional}
      
 \subsection{Conjunto de Operações de Memória}

	  \begin{functional}
		\requirement{Operação LW}
		{O módulo deve ser capaz de realizar leitura de dados na memória de dados.}
		{Essencial}
		
        \requirement{Operação SW}
        {O módulo deve ser capaz de armazenar dados na memória de dados.}
        {Essencial}

		\requirement{Escrever na memória de intruções}
		{O módulo deve ser capaz de armazenar instruções na memória de instruções.}
		{Essencial}
		
		\requirement{Ler na memória de instruções}
		{O módulo deve ser capaz de ler instruções na memória de instruções.}
		{Essencial}
	
		 
	 \end{functional}
 
          \subsection{Conjunto de Operações de Controle}
	    
	  \begin{functional}
		\requirement{Operação NOP}
		{Mneumônimo da palavra No Operation que significa uma instrução que não faz nada, a não ser refrescar as memorias dinâmicas e incrementar o contador. Esta instrução é comumente empregada para reservar espaço na memória para introduzir posteriormente uma rotina, ou para remover instruções sem precisar fazer uma nova montagem.}
		{Essencial} 
        \requirement{Operação HALT}
        {Paralisa a execução do programa até que uma requisição de interrupção seja aceita.O microprocessador não fica efetivamente parado, mas continuamente está executando a instrução HALT sem incrementar o contador de programa e com isto, o refrescamento das memórias é realizado.}
        {Essencial}
         
	  \end{functional}



  \section{Requisitos não Funcionais}
  Esta seção apresenta a lista de Requisitos não Funcionais do projeto.
\end{document}