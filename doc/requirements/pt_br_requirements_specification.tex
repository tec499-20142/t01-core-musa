%
% Portuguese-BR vertion
% 
\documentclass{article}

\usepackage{ipprocess}
% Use longtable if you want big tables to split over multiple pages.
% \usepackage{longtable}
\usepackage[utf8]{inputenc} 
% Babel package is used to translate keywors to a specific language. 
% The option "brazil" defines portuguese-brazil as default language.
% Babel is also useful for hiphenation.
\usepackage[brazil]{babel}

\sloppy

\graphicspath{{./pictures/}} % Pictures dir
\makeindex
\begin{document}

\DocumentTitle{Especificação de Requisitos}
\Project{Processador de Propósito Geral Simples}
\Organization{Universidade Estadual de Feira de Santana}
\Version{Compilação 2.1}
\capa

\newpage
	
	%%%%%%%%%%%%%%%%%%%%%%%%%%%%%%%%%%%%%%%%%%%%%%%%%%
	%% Revision History
	%%%%%%%%%%%%%%%%%%%%%%%%%%%%%%%%%%%%%%%%%%%%%%%%%%
	\section*{\center Histórico de Revisões}
	  \vspace*{1cm}
	  \begin{table}[ht]
	    \centering
	    \begin{tabular}[pos]{|m{2cm} | m{7.2cm} | m{3.8cm}|} 
	      \hline
	      \cellcolor[gray]{0.9}
	      \textbf{Data} & \cellcolor[gray]{0.9}\textbf{Descrição} & \cellcolor[gray]{0.9}\textbf{Autor(es)}\\ \hline
	      \hline
	      \small 26/09/2014 & \small Inicialização do documento & \small patrickecomp \\ \hline      
	      \small 26/09/2014 & \small Inclusão dos requisitos [FR1] à [FR8] & \small patrickecomp \\ \hline 
	      \small 28/09/2014 & \small Inclusão dos requisitos [FR12] à [FR16] & \small santana22 e gabri14el \\ \hline
	      \small 28/09/2014 & \small Concepção da introdução e inclusão dos requisitos [FR10] e [FR11] & \small manuellemacedo \\ \hline 
	      \small 29/09/2014 & \small Inclusão dos requisitos [FR17] à [FR20], Inclusão dos requisitos [NFR3] à [NFR8] & \small patrickecomp, mirelarios e tarles-w7 \\ \hline    
	      \small 29/09/2014 & \small Inclusão dos requisitos [FR21] e [FR22] & \small ardersonqdv, lucas \\ \hline   
	      \small 29/09/2014 & \small Inclusão do requisito [FR7] & \small patrickecomp \\ \hline 
	      \small 30/09/2014 & \small Refinamento da introdução,  revisão preliminar do documento e revisão da [FR16] & \small santana22 \\ \hline
	      \small 30/09/2014 & \small Revisão do requisito [FR13] & \small santana22 e gabri14el \\ \hline
	      \small 30/09/2014 & \small Adição do requisito [NFR1] & \small santana22 \\ \hline
	      \small 30/09/2014 & \small Revisão dos requisitos [FR21] e [FR22] & \small lasilva\\ \hline
	      \small 30/09/2014 & \small Adição dos requisitos [FR23] à [FR26], adição dos requisitos [NFR2], [NFR9], [NFR10]  & \small manuellemacedo \\ \hline
	      \small 05/10/2014 & \small Revisão dos requisitos [FR12] à [FR16] & \small santana22 e gabri14el \\ \hline
          
          \small 06/10/2014 & \small Adição dos requisitos [FR26] e revisão dos requisitos [NFR3], [NFR4], [NFR5] e [NFR8].  & \small patrickecomp e mirelarios \\ \hline
	      
	    \end{tabular}
	  \end{table}
	
	\newpage
	
	% TOC instantiation
	\tableofcontents
	\newpage
	
	%%%%%%%%%%%%%%%%%%%%%%%%%%%%%%%%%%%%%%%%%%%%%%%%%%
	%% Document main content
	%%%%%%%%%%%%%%%%%%%%%%%%%%%%%%%%%%%%%%%%%%%%%%%%%%
	  \section{Propósito do Documento}
  Este documento descreve a arquitetura do projeto MUSA. Como parte integrante deste documento estão os diagramas de classe, as definições de entrada e saída como também, as especificações dos circuitos internos de cada componente. Dessa forma, o principal objetivo deste documento é definir as especificações do projeto MUSA.

  \section{Stakeholders}
    \FloatBarrier
    \begin{table}[H] 
      \begin{center}
        \begin{tabular}[pos]{|m{6cm} | m{8cm}|} 
          \hline 
          \cellcolor[gray]{0.9}\textbf{Nome} & \cellcolor[gray]{0.9}\textbf{Papel/Responsabilidades} \\  
           \hline Manuelle 	& Gerência \\
           \hline Manuelle, Patrick, Lucas, Mirela, Vinícius, Gabriel, Weverson, Anderson e Tarles 	& Análise \\
           \hline Manuelle, Patrick, Lucas, Mirela, Vinícius, Gabriel, Weverson, Anderson e Tarles 	& Projeto \\ \hline
        \end{tabular}
      \end{center}
    \end{table} 

\section{Visão Geral do Documento}

O presente documento é apresentado como segue:

  \begin{itemize}
   \item \textbf{Capítulo 2 --} Este capítulo apresenta uma visão geral da arquitetura, com foco nas entradas e saídas do sistema e na arquitetura geral do mesmo;
   \item \textbf{Capítulo 3 --} Este capítulo descreve a arquitetura interna do IP a partir do detalhamento dos seus componentes, da definição das portas de entrada e saída e das especificações do caminho de dados (\textit{datapath}).
  \end{itemize}

  % inicio da tabela de acronimos e abreviacoes do documento
  \section{Acrônimos e Abreviações}
    \FloatBarrier
    \begin{table}[H]
      \begin{center}
        \begin{tabular}[pos]{|m{2cm} | m{12cm}|} 
          \hline
          \cellcolor[gray]{0.9}\textbf{Sigla} & \cellcolor[gray]{0.9}\textbf{Descrição} \\ \hline
          PC      &  Program Counter  \\ \hline
          OPCODE  &  Operation Code  \\ \hline
          RF  &  Register Flag  \\ \hline
          CST  &  Constant  \\ \hline
        \end{tabular}
      \end{center}
    \end{table}  
  % fim    
	
	\subsection{Visão Geral do Documento}
	  \begin{itemize}
	   \item \textbf{Requisitos funcionais -} lista de todos os requisitos funcionais.
	   \item \textbf{Requisitos não funcionais -} lista de todos os requisitos não funcionais.
	%   \item \textbf{Dependências -} conjunto de dependências de IP-cores previstos.
	   % \item \textbf{Notas -} apresenta a lista de notas apresentadas ao longo do documento.
	   % \item \textbf{Referências -} lista de todos os textos referenciados nesse documento.
	  \end{itemize}
	
	  \subsection{Definições}
	  
	  \FloatBarrier
	  \begin{table}[H]
	    \begin{center}
	      \begin{tabular}[pos]{|m{2.5cm} | m{11.5cm}|} 
	        \hline
	        \cellcolor[gray]{0.9}\textbf{Termo} & \cellcolor[gray]{0.9}\textbf{Descrição} \\ \hline
	        \flushright \centering {Requisito Funcional} & Requisitos de hardware que compõem os módulos, descrevendo as ações que o 
	                                    mesmo deve estar apto a executar. Estas informações são capturadas a partir 
	                                    do desenvolvimento dos casos de uso, que documentam as entradas, os processos 
	                                    e as saídas geradas.  \\ \hline
	        \flushright \centering {Requisito Não \mbox{Funcional}} & Requisitos de hardware que compõem os módulos, representando as características 
	                                    que o mesmo deve ter, ou restrições que o mesmo deve operar. Estas características
	                                    referem-se a técnicas, algoritmos, tecnologias e especificidades do Sistema como um todo.  \\ \hline
	%        \flushright \centering {Dependências}              & Requisitos de reuso de IP-cores, descrevendo as funções que cada um deve exercer. \\ \hline
	      \end{tabular}
	    \end{center}
	  \end{table}  
	
	  % inicio da tabela de acronimos e abreviacoes do documento
	  \subsection{Acrônimos e Abreviações}
	    \FloatBarrier
	    \begin{table}[H]
	      \begin{center}
	        \begin{tabular}[pos]{|m{2cm} | m{12cm}|} 
	          \hline
	          \cellcolor[gray]{0.9}\textbf{Sigla} & \cellcolor[gray]{0.9}\textbf{Descrição} \\ \hline
	          FR      & Requisito Funcional  \\ \hline
	          NFR     & Requisito Não Funcional  \\ \hline
	%          D       & Dependência  \\ \hline
			  PC      & Program Counter \\ \hline 
	        \end{tabular}
	      \end{center}
	    \end{table}  
	  % fim  
	
	  % inicio da descriao de prioridades de requisitos
	  \subsection{Prioridades dos Requisitos}
	    \FloatBarrier
	    \begin{table}[H]
	      \begin{center}
	        \begin{tabular}[pos]{|m{2cm} | m{12cm}|} 
	          \hline
	          \cellcolor[gray]{0.9}\textbf{Prioridade} & \cellcolor[gray]{0.9}\textbf{Característica} \\ \hline
	          Importante      & Requisito sem o qual o sistema funciona, porém não como deveria.  \\ \hline
	          Essencial       & Requisito que deve ser implementado para que o sistema funcione.  \\ \hline
	          Desejável       & Requisito que não compromete o funcionamento do sistema.  \\ \hline
	        \end{tabular}
	      \end{center}
	    \end{table}  
	  % fim
	
	\section{Requisitos Funcionais}
	
	\subsection{Conjunto de Instruções Aritméticas} 
  
    \begin{functional}
      \requirement{Instrução ADD}
      {O módulo deve ser capaz de realizar a operação de soma de dois valores de 32 bits.}
      {Essencial}
      
       \requirement{Instrução ADDI}
      {O módulo deve ser capaz de realizar a operação de soma de dois valores de 32 bits, sendo um destes valor imediato.}
      {Essencial}

      \requirement{Instrução SUB}
      {O módulo deve ser capaz de realizar a operação de subtração de dois valores de 32 bits.}
      {Essencial}
      
      \requirement{Instrução SUBI}
      {O módulo deve ser capaz de realizar a operação de subtração de dois valores de 32 bits, sendo um destes valor imediato.}
      {Essencial}

      \requirement{Instrução MUL}
      {O módulo deve ser capaz de realizar a operação de multiplicação de dois valores de 32 bits. }
      {Essencial}

      \requirement{Instrução DIV}
      {O módulo deve ser capaz de realizar a operação de divisão de dois valores de 32 bits.}
      {Essencial} 
     

      \requirement{Tamanho da palavra de saída}
      {O componente deve apresentar uma saída única de 32 bits para todas as operações aritméticas.}
      {Essencial}       

      \requirement{Detecção de overflow aritmético}
      {O módulo deve ser capaz de detectar \textit{overflow} aritmético.}
      {Essencial}
      
      \requirement{Detecção de underflow aritmético}
      {O módulo deve ser capaz de detectar \textit{underflow} aritmético.}
      {Essencial}
      
%      \requirement{Detecção de zero}
%      {O módulo deve ser capaz de detectar no resultado da operação o valor zero e habilitar uma \textit{flag}.}
%      {Importante}
      
    \end{functional}
	
	  \subsection{Conjunto de Operações Lógicas} 
  
    \begin{functional}
    
      \requirement{Operação AND}
      {O módulo deve ser capaz de realizar a operação AND lógico de dois operandos de 8 bits.}
      {Essencial}

      \requirement{Operação OR}
      {O módulo deve ser capaz de realizar a operação OR lógico de dois operandos de 8 bits.}
      {Essencial}  
      
      \requirement{Operação NOT}
      {O módulo deve ser capaz de realizar a operação NOT lógico de um operando de 8 bits.}
      {Essencial}   
      
      \requirement{Operação CMP}
      {O módulo deve ser capaz de comparar dois registradores de 8 bits cada e ativar um \textit{flag}.}
      {Essencial}
      
    \end{functional}  
	    
	\subsection{Conjunto de Operações de Desvio}
	    
	  \begin{functional}
		\requirement{Operação JR}
		{O módulo deve ser capaz de realizar desvios para endereços específicos na memória.}
		{Essencial} 
		 
		 \requirement{Operação JPC}
		 {O módulo deve ser capaz de realizar desvios para endereços relativos ao PC.}
		 {Essencial}
		 
		 \requirement{Operação BRFL}
		 {O módulo deve ser capaz de realizar desvios condicionais com base em um \textit{flag}.}
		 {Essencial}
		 
		 \requirement{Operação CALL}
		 {O módulo deve ser capaz de desviar a execução para uma sub-rotina.}
		 {Essencial}
		 
		 \requirement{Operação RET}
		 {O módulo deve ser capaz de retornar de uma sub-rotina.}
		 {Essencial}
         
	  \end{functional}
	      
	\subsection{Conjunto de Instruções de acesso à Memória}

	  \begin{functional}
		\requirement{Operação LW}
		{O módulo deve ser capaz de realizar leitura de dados na memória de dados.}
		{Essencial}
		
		\requirement{Operação SW}
		{O módulo deve ser capaz de armazenar dados na memória de dados.}
		{Essencial}
		
		\requirement{Leitura da memória de instruções}
		{O módulo deve ser capaz de ler instruções na memória de instruções.}
		{Essencial}
                
        \requirement{Modo de endereçamento de dados}
        {O módulo deve possuir três modos de endereçamento: imediato, por registrador e base-deslocamento.}
        {Essencial}

		\requirement{Padrão de conversão de tamanho de palavra}
		{O módulo deve converter dados de 16 bits para 32 bits utilizando o padrão zero-extend ou o padrão sign-extend.}
		{Essencial}
		 
	 \end{functional}
	 
	\subsection{Conjunto de Operações de Controle}
	    
	\begin{functional}
		
		\requirement{Operação NOP}
		{O módulo não realiza operação. \textit{Refresh} nas memórias e incrementar o contador de programa. Reserva espaço de memória para carregar novas instruções.}
		{Essencial} 
       
        \requirement{Operação HALT}
        {O módulo paralisa a execução de um programa e continua a realizar operação HALT até que uma requisição de interrupção seja aceita. Não incrementa o contador de programa. \textit{Refresh} das memórias é realizado.}
        {Essencial}
 		
	  \end{functional}
		  
	 \section{Requisitos não Funcionais}
	  Esta seção apresenta a lista de Requisitos não Funcionais do projeto.
	  
	  \begin{nonfunctional}
	  	
	  \requirement{Interfaceamento}
	  {Este módulo deve ser capaz de interfacear com um mecanismo de saída.}
	  {Desejável}
	  
	  \requirement{Tamanho da palavra de instrução}
	  {O tamanho da palavra de intrução será de 32 bits.}
	  {Importante}
	  
	   \requirement{Modo de Armazenamento}
        {O tratamento e o armazenamento na memória de dados será feito no formato big-endian.}
        {Importante}
        
        \requirement{Modo de endereçamento da memória de dados e instruções}
        {O módulo deve possuir três modos de endereçamento: imediato, por registrador e base-deslocamento.}
        {Importante}
        
       \requirement{Tipo de memória de dados}
        {O módulo deve utilizar memória do tipo SPRAM.}
        {Imnportante}
        
       \requirement{Tamanho da memória de dados}
        {A memória de dados deve ser de 512 Kbyte.}
        {Essencial}
        
       \requirement{Tamanho da memória de instruções}
        {A memória de dados deve ser de 512 Kbyte.}
        {Essencial}
        
       \requirement{Tamanho do banco de registradores}
        {O banco de registradores possui 32 registradores cada um com 32 bits.}
        {Essencial}
	  
		\requirement{Formato dos Números}
		{Na representação dos números inteiros para a soma e subtração será adotado números com sinal e o complemento a 2 para representação de valores negativos}
		{Importante}
		
		\requirement{Arquitetura multi-ciclo}
		{As instruções serão executadas em alguns ciclos de clock.}
		{Essencial}		
			  
	  \end{nonfunctional}
	  
	\end{document}