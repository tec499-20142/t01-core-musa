\section{Introdução}

Em um ambiente empresarial competitivo e dinâmico, possuir o domínio de técnicas que possam agilizar e minimizar as atividades de produção são necessárias para se obter êxito. Principalmente, quando o segmento de \textit{Systems on Chip} cresce rapidamente em todo o mundo e, essa conjuntura, necessita de um rápido desenvolvimento de novas e mais acessíveis tecnologias. Dessa maneira, torna-se quase uma urgência a formação de mão de obra especializada nesse (e em outros) segmentos. Nesse contexto, a \textbf{Fazemos Qualquer Negócio Inc.} em parceria com a \textbf{Universidade Estadual de Feira de Santana}, foi contratada para o desenvolvimento de um \textit{IP-Core} de um micro processador de propósito geral que será utilizado em escolas da África com o intuito de impulsionar o desenvolvimento deste continente. \\
A seguir, como parte da fase \textbf{Concepção}, este documento apresenta o levantamento de diversos requisitos identificados a partir das necessidades dos nossos clientes para produção deste \textit{IP-Core}.