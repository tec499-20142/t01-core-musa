\section{Estágio de execução}
	\subsection{Diagrama de Classe}
  \begin{figure}[h]
    \begin{center}
	\begin{tikzpicture}
	\umlclass[x=0,y=0]{EX}{
	+ data_a : input bit[32] \\
	+ data_b : input bit[32] \\ 
	+ pc_in : input bit[32] \\ 
	+ orig_pc: input bit [2] \\
	+ op_alu : input bit[5] \\
	+ origin_aalu : input bit[1]\\
	+ origin_balu : input bit[2] \\
	+ instruction : output bit[32] \\
	+ offset_inst : input [32] \\}			
	{+<<comb>> calc_next_pc() \\
	+ <<comb>> cal_al_operation() \\}
	\end{tikzpicture}
\end{center}

  \end{figure}
		
		\subsection{Definições de entrada e saída}
		
	\begin{center}
		\begin{longtable}[pos]{| l | c | c | m{7cm} |} \hline
			\multicolumn{1}{|c|}{\cellcolor[gray]{0.9}\textbf{Nome}} & 
			\multicolumn{1}{c|}{\cellcolor[gray]{0.9}\textbf{Tamanho}} & 
			\multicolumn{1}{c|}{\cellcolor[gray]{0.9}\textbf{Direção}} &
			\multicolumn{1}{c|}{\cellcolor[gray]{0.9}\textbf{Descrição}} \\ \hline
			\endhead
			\hline
			\endlastfoot

			clock & 1 & Entrada & Clock do sistema \\ \hline
			reset & 1 & Entrada & Sinal de reinício do estágio\\ \hline
			data\_a & 32 & Entrada & Dado do primeiro operando \\ \hline
			data\_b & 32 & Entrada & Dado do segundo operando \\ \hline
			pc\_in & 32 & Entrada & Valor do PC atual \\ \hline
			pc\_1 & 32 & Entrada & Valor de PC + 1\\ \hline
			immediate & 32 & Entrada & Constante numérica  \\ \hline
			immediate\_div\_4 & 32 & Entrada & Contante numérica \\\hline
			stack & 32 & Entrada & Valor do PC proveniente da Pilha de Instruções \\\hline
			data\_a\_select & 2 & Entrada & Sinal para selecionar o conteúdo do primeiro operando \\\hline
			data\_b\_select & 2 & Entrada & Sinal para selecionar o conteúdo do segundo operando operando \\\hline
			pc\_select & 2 & Entrada & Sinal para selecionar o endereço da próxima instrução a ser executada \\\hline
			alu\_control & 2 & Entrada & Sinal informa qual o tipo da operação que será realizada\\\hline
			func & 6 & Saída & Variante especifica da operação \\\hline
			result & 32 & Saída & Resultado da operação \\\hline
			flag & 3 & Saída & Informações sobre o resultado da operação \\\hline
			next\_pc & 32 & Saída & Endereço da próxima instrução a ser executada\\\hline
			
			%rt\_data\_mem & 5 & Entrada & Dado vindo da memoria \\ \hline
			%opcode & 5 & Entrada & Código da operação\\ \hline
			%control\_signal & TBD & Entrada & Sinal de comando da unidade de controle. \\ \hline
			%alu\_result & 32 & Saída & Representação do resultado da operação. \\ \hline
			%pc\_out & 32 & Saída & Valor do PC atual. \\ \hline
			%write\_reg\_mem & 5 & Saída & Sinal proveniente da UC que habilita a escrita no registrador. \\ \hline
			%memory\_address & 32 & Saída & Endereço da memória. \\ \hline
			%memory\_write\_enable & TBD & Saída & Sinal proveniente da UC que habilita a escrita na memória.

		\end{longtable}
	\end{center}
	
\newpage	
	
	\subsection{Datapath Interno}
	
	\begin{figure}[ht]
		\begin{center}
		\includegraphics[scale = 0.8]{./datapath/stage3.eps}
		\end{center}
	\end{figure}
