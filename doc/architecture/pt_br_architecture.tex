%
% Portuguese-BR vertion
% 
\documentclass{report}

\usepackage{ipprocess}
% Use longtable if you want big tables to split over multiple pages.
\usepackage{longtable}
\usepackage[utf8]{inputenc} 
\usepackage[brazil]{babel} % Uncomment for portuguese
\usepackage{pdflscape} % set ladscape/portrait pdf pages
\usepackage{tikz}
\usepackage{tikz-uml}
\usepackage{multirow}

\sloppy

\graphicspath{{./pictures/}} % Pictures dir
\makeindex
\begin{document}


%%%%%%%%%%%%%%%%%%%%%%%%%%%%%%%%%%%%%%%%%%%%%%%%%%
%% Building front cover
%%%%%%%%%%%%%%%%%%%%%%%%%%%%%%%%%%%%%%%%%%%%%%%%%%
\DocumentTitle{Documento de Arquitetura}
\Project{MUSA}
\Organization{Fazemos Qualquer Negócio Inc.}
\Version{Compilação 2.0}
% Make front cover
\capa
%

%%%%%%%%%%%%%%%%%%%%%%%%%%%%%%%%%%%%%%%%%%%%%%%%%%
%% Revision History
%%%%%%%%%%%%%%%%%%%%%%%%%%%%%%%%%%%%%%%%%%%%%%%%%%
\chapter*{Histórico de Revisões}
  \vspace*{1cm}
  \begin{table}[ht]
    \centering
    \begin{tabular}[pos]{|m{2cm} | m{8cm} | m{4cm}|} 
      \hline
      \cellcolor[gray]{0.9}
      \textbf{Date} & \cellcolor[gray]{0.9}\textbf{Descrição} & \cellcolor[gray]{0.9}\textbf{Autor(s)}\\ \hline
      25/06/2014 & Concepção do documento & joaocarlos \\ \hline
      15/10/2014 & Adição da subseção de acesso à memória & Weverson Gomes \\ \hline
      16/10/2014 & Adição da subseção de acesso à memória & Weverson Gomes \\ \hline
      16/10/2014 & Adição da seção "Leitura da Instrução" com dados preliminares e modificação do nome do projeto no documento. & santana22 e gabri14el.\\ \hline
      19/10/2014 & Modificações na seção "Leitura da Instrução" & santana22 \\ \hline
      20/10/2014 & Adição da tabela de descrição das instruções & manuellemacedo \\ \hline
    \end{tabular}
  \end{table}

% TOC instantiation
\tableofcontents

%%%%%%%%%%%%%%%%%%%%%%%%%%%%%%%%%%%%%%%%%%%%%%%%%%
%% Document main content
%%%%%%%%%%%%%%%%%%%%%%%%%%%%%%%%%%%%%%%%%%%%%%%%%%
\chapter{Introdução}
  
	  \section{Propósito do Documento}
  Este documento descreve a arquitetura do projeto MUSA. Como parte integrante deste documento estão os diagramas de classe, as definições de entrada e saída como também, as especificações dos circuitos internos de cada componente. Dessa forma, o principal objetivo deste documento é definir as especificações do projeto MUSA.

  \section{Stakeholders}
    \FloatBarrier
    \begin{table}[H] 
      \begin{center}
        \begin{tabular}[pos]{|m{6cm} | m{8cm}|} 
          \hline 
          \cellcolor[gray]{0.9}\textbf{Nome} & \cellcolor[gray]{0.9}\textbf{Papel/Responsabilidades} \\  
           \hline Manuelle 	& Gerência \\
           \hline Manuelle, Patrick, Lucas, Mirela, Vinícius, Gabriel, Weverson, Anderson e Tarles 	& Análise \\
           \hline Manuelle, Patrick, Lucas, Mirela, Vinícius, Gabriel, Weverson, Anderson e Tarles 	& Projeto \\ \hline
        \end{tabular}
      \end{center}
    \end{table} 

\section{Visão Geral do Documento}

O presente documento é apresentado como segue:

  \begin{itemize}
   \item \textbf{Capítulo 2 --} Este capítulo apresenta uma visão geral da arquitetura, com foco nas entradas e saídas do sistema e na arquitetura geral do mesmo;
   \item \textbf{Capítulo 3 --} Este capítulo descreve a arquitetura interna do IP a partir do detalhamento dos seus componentes, da definição das portas de entrada e saída e das especificações do caminho de dados (\textit{datapath}).
  \end{itemize}

  % inicio da tabela de acronimos e abreviacoes do documento
  \section{Acrônimos e Abreviações}
    \FloatBarrier
    \begin{table}[H]
      \begin{center}
        \begin{tabular}[pos]{|m{2cm} | m{12cm}|} 
          \hline
          \cellcolor[gray]{0.9}\textbf{Sigla} & \cellcolor[gray]{0.9}\textbf{Descrição} \\ \hline
          PC      &  Program Counter  \\ \hline
          OPCODE  &  Operation Code  \\ \hline
          RF  &  Register Flag  \\ \hline
          CST  &  Constant  \\ \hline
        \end{tabular}
      \end{center}
    \end{table}  
  % fim

\chapter{Visão Geral da Arquitetura}

	\section{Restrições}
	\begin{itemize}
	\item \textbf{Restrições --} %fazer junto com os desenvolvedores
	\end{itemize}
	
	\section{Codificação das instruções}
	Instrução é uma palavra da linguagem de máquina, sua codificação é de fundamental importância para o processamento das operações.	 
  	\\Todas as instruções contém 32 bits. Exitem 4 formatos de instruções: R (R-type), I (I-type), Load/Store e Jump.
  	\\O formato R está relacionado as instruções lógicas e aritméticas.
	\begin{figure}[H]
    	\centering
    	\includegraphics{r-format}
    	\caption{Formato R}
		\label{r_format}
	\end{figure}
	Seus respectivos campos são:
	\begin{itemize}
	\item \textbf{OPCODE} - Código da operação básica da instrução.
	\item \textbf{RS} - Registrador do primeiro operando de origem.
	\item \textbf{RT} - Registrador do segundo operando de origem.
	\item \textbf{RD} - Registrador destino.
	\item \textbf{SHAMT} - \textit{Shift amount}; Quantidade de deslocamento.
	\item \textbf{FUNCT} - Função; Esse campo seleciona a variante específica da operação no campo opcode, e as vezes, é chamado de código de função.
\end{itemize}	  	
  	Um segundo tipo de formato de instrução é chamado de formato I, utilizado pelas instruções imediatas e de transferência de dados.
	\begin{figure}[H]
    	\centering
    	\includegraphics{i-format}
    	\caption{Formato I}
		\label{i_format}
  	\end{figure}
Seus respectivos campos são:
	\begin{itemize}
	\item \textbf{OPCODE} - Código da operação básica da instrução.
	\item \textbf{RS} - Registrador do operando de origem.
	\item \textbf{RT} - Registrador destino.
	\item \textbf{ADDRESS} - Endereço de memória ou constante numérica.
\end{itemize}	  	
  	
  	\begin{figure}[H]
    	\centering
    	\includegraphics{load-store}
    	\caption{Formato Load/Store}
		\label{loadstore}
  	\end{figure}
  	
   	\begin{figure}[H]
    	\centering
    	\includegraphics{jump}
    	\caption{Formato Jump}
		\label{jump}
  	\end{figure}
  	
 
	
	\section{Descrição dos Componentes}
  A unidade de processamento a ser desenvolvida é composta a partir dos seguintes componentes:

  \begin{itemize}
    \item \textbf{Serial Controller --} Controlador para comunicação com módulo de transmissão serial através do protocolo RS232.
    \item \textbf{Interface Control --} Interface de controle, responsável por fazer a leitura correta das informações da serial e transmiti-las para a unidade de processamento.
    \item \textbf{Processing Unit --} Unidade responsável pela realização das operações e armazenamento do resultado.
  \end{itemize}

  \section{Diagrama de Classe (Interface)}
  \begin{figure}[H]
    \centering
      \section{Interface de Comunicação}

    \subsection{Diagrama de Classe}
      \begin{figure}[H]
        \centering
        \input{./classes/if_ctrl.tikz}
      \end{figure}

    \subsection{Definições de Entrada e Saída}
      \FloatBarrier
      \begin{center}
        \begin{longtable}[pos]{| l | c | c | m{7cm} |} \hline         
          \multicolumn{1}{|c|}{\cellcolor[gray]{0.9}\textbf{Nome}} & 
          \multicolumn{1}{c|}{\cellcolor[gray]{0.9}\textbf{Tamanho}} & 
          \multicolumn{1}{c|}{\cellcolor[gray]{0.9}\textbf{Direção}} &
          \multicolumn{1}{c|}{\cellcolor[gray]{0.9}\textbf{Descrição}} \\ \hline
          \endfirsthead
          \hline
          \multicolumn{4}{|l|}%
          {{\bfseries continuação da página anterior}} \\
          \hline
          \multicolumn{1}{|c|}{\cellcolor[gray]{0.9}\textbf{Nome}} & 
          \multicolumn{1}{c|}{\cellcolor[gray]{0.9}\textbf{Tamanho}} & 
          \multicolumn{1}{c|}{\cellcolor[gray]{0.9}\textbf{Direção}} &
          \multicolumn{1}{c|}{\cellcolor[gray]{0.9}\textbf{Descrição}} \\ \hline
          \endhead

          \multicolumn{4}{|r|}{{continua na próxima página}} \\ \hline
          \endfoot

          \hline
          \endlastfoot

          clock\_in                & 1   & entrada   & Clock principal do sistema.    \\ \hline
          reset\_in                & 1   & entrada   & Sinal de reset geral do sistema.    \\ \hline
          rx\_data\_ready\_in      & 1   & entrada   & Indica que o dado foi recebido pelo controle RS232.    \\ \hline
          rx\_data\_in             & 8   & entrada   & Dado proveniente da transmissão.    \\ \hline
          data\_a\_out             & 8   & saída   & Dado do primeiro operando.    \\ \hline
          data\_b\_out             & 8   & saída   & Dado do segundo operando.    \\ \hline
          operation\_out          & TBD   & saída   & Código da operação.    \\ 
        \end{longtable}
      \end{center}    

    %\subsection{Datapath Interno}

    \subsection{Máquina de Estados}
      \begin{figure}[H]
        \centering
        \begin{tikzpicture} 
        \umlstateinitial[name=idle]
        \umlstatefinal[x=8, y=-3.2, name=final]
        \umlbasicstate[x=0, y=-4, do=counter++, name=read]{read\_data}  
        \umlbasicstate[x=4, y=-3.5, name=send]{send\_data}    

        \umltrans[arg={rx\_data\_ready\_in}]{idle}{read}
        \umltrans{read}{send}
        \umltrans{send}{final}
        \umltrans[recursive=-160|-120|3cm, recursive direction=right to bottom, arg={counter != 3}, pos=1.5]{read}{read} 
        \end{tikzpicture}    
      \end{figure}

    \begin{landscape}
      \subsection{Diagrama de Temporização}
        \begin{figure}[H]
          \centering
          \includegraphics[width=\linewidth]{timing/comunication_timing.eps}
        \end{figure}
    \end{landscape}
  \end{figure}

  \section{Definições de Entrada e Saída}
  \FloatBarrier
    \begin{center}
      \begin{longtable}[pos]{| l | c | c | m{7cm} |} \hline         
        \multicolumn{1}{|c|}{\cellcolor[gray]{0.9}\textbf{Nome}} & 
        \multicolumn{1}{c|}{\cellcolor[gray]{0.9}\textbf{Tamanho}} & 
        \multicolumn{1}{c|}{\cellcolor[gray]{0.9}\textbf{Direção}} &
        \multicolumn{1}{c|}{\cellcolor[gray]{0.9}\textbf{Descrição}} \\ \hline
        \endfirsthead
        \hline
        \multicolumn{4}{|l|}%
        {{\bfseries continuação da página anterior}} \\
        \hline
        \multicolumn{1}{|c|}{\cellcolor[gray]{0.9}\textbf{Nome}} & 
        \multicolumn{1}{c|}{\cellcolor[gray]{0.9}\textbf{Tamanho}} & 
        \multicolumn{1}{c|}{\cellcolor[gray]{0.9}\textbf{Direção}} &
        \multicolumn{1}{c|}{\cellcolor[gray]{0.9}\textbf{Descrição}} \\ \hline
        \endhead

        \multicolumn{4}{|r|}{{continua na próxima página}} \\ \hline
        \endfoot

        \hline
        \endlastfoot

        clock\_in                & 1   & entrada   & Clock principal do sistema.    \\ \hline
        reset\_in                & 1   & entrada   & Sinal de reset geral do sistema.    \\ \hline
        rx\_in                   & 1   & entrada   & Dado serial da RS232. \\ \hline
        % tx\_out                  & 1   & saída     & Dado serial RS232 a ser transmitido. \\ \hline
        result\_data\_out        & 8   & saída     & Representação do resultado da operação. \\ \hline
        overflow\_out            & 1   & saída     & Sinal indicador de overflow aritmético. \\
      \end{longtable}
    \end{center} 
  \section{Datapath Interno}
    \begin{figure}[H]
      \centering
      \includegraphics[width=\linewidth]{datapath/ip_datapath.eps}
    \end{figure}

% inicio das descrições de arquitetura para cada componente do sistema
\chapter{Descrição da Arquitetura}

	  \section{Unidade de Processamento}

    \subsection{Diagrama de Classe}
      \begin{figure}[H]
      \centering
      \input{./classes/processing_unit.tikz}
    \end{figure}

    \subsection{Definições de Entrada e Saída}
      \FloatBarrier
      \begin{center}
        \begin{longtable}[pos]{| l | c | c | m{7cm} |} \hline         
          \multicolumn{1}{|c|}{\cellcolor[gray]{0.9}\textbf{Nome}} & 
          \multicolumn{1}{c|}{\cellcolor[gray]{0.9}\textbf{Tamanho}} & 
          \multicolumn{1}{c|}{\cellcolor[gray]{0.9}\textbf{Direção}} &
          \multicolumn{1}{c|}{\cellcolor[gray]{0.9}\textbf{Descrição}} \\ \hline
          \endfirsthead
          \hline
          \multicolumn{4}{|l|}%
          {{\bfseries continuação da página anterior}} \\
          \hline
          \multicolumn{1}{|c|}{\cellcolor[gray]{0.9}\textbf{Nome}} & 
          \multicolumn{1}{c|}{\cellcolor[gray]{0.9}\textbf{Tamanho}} & 
          \multicolumn{1}{c|}{\cellcolor[gray]{0.9}\textbf{Direção}} &
          \multicolumn{1}{c|}{\cellcolor[gray]{0.9}\textbf{Descrição}} \\ \hline
          \endhead

          \multicolumn{4}{|r|}{{continua na próxima página}} \\ \hline
          \endfoot

          \hline
          \endlastfoot

          clock\_in                & 1   & entrada   & Clock principal do sistema.    \\ \hline
          reset\_in                & 1   & entrada   & Sinal de reset geral do sistema.    \\ \hline
          data\_a\_in              & 8   & entrada   & Dado do primeiro operando.    \\ \hline
          data\_b\_in              & 8   & entrada   & Dado do segundo operando.    \\ \hline
          operation\_in            & TBD   & entrada   & Código da operação.    \\ \hline
          result\_data\_out        & 8   & saída     & Representação do resultado da operação. \\ \hline
          overflow\_out            & 1   & saída     & Sinal indicador de overflow aritmético. \\
        \end{longtable}
      \end{center} 

    \subsection{Datapath Interno}
      \begin{figure}[H]
        \centering
        \includegraphics[width=\linewidth]{datapath/processing_datapath.eps}
      \end{figure}
    \newpage
	
	\begin{center}
	\begin{tikzpicture}
		\umlclass[x=0,y=0]{Instruction Fetch}{
				+ clock : input bit \\
				+ pcInput : input bit[32] \\ 
				+ pcWrite : input bit \\
				+ pcOutput : output bit[32] \\
				+ instruction : output bit[32] \\
			}			
			{
			
			}
	\end{tikzpicture}
\end{center}
	
	\newpage
	
	\begin{center}
	\begin{tikzpicture}
	\umlclass[x=0,y=0]{Acess Memory}{
	+ clock : input bit \\
	+ reset : input bit \\ 
	+ address : input bit[32] \\ 
	+ memRead : input bit \\ 
	+ memWrite : input bit\\ 
	+ memToReg : input bit \\
	+ data\_in : input bit[32] \\ 
	+ writeBack : output bit[32]\\}
	{\underline{+<<comb>> accessMem()}\\
	+<<comb>> writeReg()}
	\end{tikzpicture}
\end{center}

	  \section{Interface de Comunicação}

    \subsection{Diagrama de Classe}
      \begin{figure}[H]
        \centering
        \input{./classes/if_ctrl.tikz}
      \end{figure}

    \subsection{Definições de Entrada e Saída}
      \FloatBarrier
      \begin{center}
        \begin{longtable}[pos]{| l | c | c | m{7cm} |} \hline         
          \multicolumn{1}{|c|}{\cellcolor[gray]{0.9}\textbf{Nome}} & 
          \multicolumn{1}{c|}{\cellcolor[gray]{0.9}\textbf{Tamanho}} & 
          \multicolumn{1}{c|}{\cellcolor[gray]{0.9}\textbf{Direção}} &
          \multicolumn{1}{c|}{\cellcolor[gray]{0.9}\textbf{Descrição}} \\ \hline
          \endfirsthead
          \hline
          \multicolumn{4}{|l|}%
          {{\bfseries continuação da página anterior}} \\
          \hline
          \multicolumn{1}{|c|}{\cellcolor[gray]{0.9}\textbf{Nome}} & 
          \multicolumn{1}{c|}{\cellcolor[gray]{0.9}\textbf{Tamanho}} & 
          \multicolumn{1}{c|}{\cellcolor[gray]{0.9}\textbf{Direção}} &
          \multicolumn{1}{c|}{\cellcolor[gray]{0.9}\textbf{Descrição}} \\ \hline
          \endhead

          \multicolumn{4}{|r|}{{continua na próxima página}} \\ \hline
          \endfoot

          \hline
          \endlastfoot

          clock\_in                & 1   & entrada   & Clock principal do sistema.    \\ \hline
          reset\_in                & 1   & entrada   & Sinal de reset geral do sistema.    \\ \hline
          rx\_data\_ready\_in      & 1   & entrada   & Indica que o dado foi recebido pelo controle RS232.    \\ \hline
          rx\_data\_in             & 8   & entrada   & Dado proveniente da transmissão.    \\ \hline
          data\_a\_out             & 8   & saída   & Dado do primeiro operando.    \\ \hline
          data\_b\_out             & 8   & saída   & Dado do segundo operando.    \\ \hline
          operation\_out          & TBD   & saída   & Código da operação.    \\ 
        \end{longtable}
      \end{center}    

    %\subsection{Datapath Interno}

    \subsection{Máquina de Estados}
      \begin{figure}[H]
        \centering
        \begin{tikzpicture} 
        \umlstateinitial[name=idle]
        \umlstatefinal[x=8, y=-3.2, name=final]
        \umlbasicstate[x=0, y=-4, do=counter++, name=read]{read\_data}  
        \umlbasicstate[x=4, y=-3.5, name=send]{send\_data}    

        \umltrans[arg={rx\_data\_ready\_in}]{idle}{read}
        \umltrans{read}{send}
        \umltrans{send}{final}
        \umltrans[recursive=-160|-120|3cm, recursive direction=right to bottom, arg={counter != 3}, pos=1.5]{read}{read} 
        \end{tikzpicture}    
      \end{figure}

    \begin{landscape}
      \subsection{Diagrama de Temporização}
        \begin{figure}[H]
          \centering
          \includegraphics[width=\linewidth]{timing/comunication_timing.eps}
        \end{figure}
    \end{landscape}

% Optional bibliography section
% To use bibliograpy, first provide the ipprocess.bib file on the root folder.
% \bibliographystyle{ieeetr}
% \bibliography{ipprocess}

\end{document}
