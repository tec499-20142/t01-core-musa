%
% Portuguese-BR version
%

\documentclass{report}

\usepackage{ipprocess}
% Use longtable if you want big tables to split over multiple pages.
\usepackage{longtable}
\usepackage[utf8]{inputenc} 
\usepackage[brazil]{babel} % Uncomment for portuguese
\usepackage{pdflscape} % set ladscape/portrait pdf pages
\usepackage{tikz}
\usepackage{tikz-uml}
\usepackage{multirow}
\usepackage{graphicx}

\sloppy

\graphicspath{{./pictures/}} % Pictures dir
\makeindex
\begin{document}


%%%%%%%%%%%%%%%%%%%%%%%%%%%%%%%%%%%%%%%%%%%%%%%%%%
%% Building front cover
%%%%%%%%%%%%%%%%%%%%%%%%%%%%%%%%%%%%%%%%%%%%%%%%%%
\DocumentTitle{Documento de Arquitetura}
\Project{MUSA}
\Organization{Fazemos Qualquer Negócio Inc.}
\Version{Compilação 3.1}
% Make front cover
\capa
%

%%%%%%%%%%%%%%%%%%%%%%%%%%%%%%%%%%%%%%%%%%%%%%%%%%
%% Revision History
%%%%%%%%%%%%%%%%%%%%%%%%%%%%%%%%%%%%%%%%%%%%%%%%%%
\section*{\center Histórico de Revisões}
	\vspace*{1cm}
	\begin{table}[h!]
		\centering
		\begin{longtable}[pos]{|m{2cm} | m{7.2cm} | m{3.8cm}|} \hline \cellcolor[gray]{0.9}

\textbf{Data} & \cellcolor[gray]{0.9}\textbf{Descrição} & \cellcolor[gray]{0.9} \textbf{Autor(es)}\\ \hline
      \small 25/06/2014 & \small Concepção do documento & \small joaocarlos \\ \hline
      \small 15/10/2014 & \small Adição da subseção de acesso à memória & \small weversongomes \\ \hline
      \small 16/10/2014 & \small Adição da seção "Leitura da Instrução" e modificação do nome do projeto. & \small santana22 e gabri14el.\\ \hline
      \small 19/10/2014 & \small Modificações na seção "Leitura da Instrução" & \small santana22 \\ \hline
      \small 20/10/2014 & \small Correções na subseção de acesso à memória & \small weversongomes \\ \hline
      \small 25/10/2014 & \small Adição das descrições da codificação & \small manuellemacedo \\ \hline
      \small 29/10/2014 & \small Unificação das subseções de acesso à memória com a de write back & \small weversongomes \\ \hline
      \small 29/10/2014 & \small Adição das descrições dos componentes & \small manuellemacedo \\ \hline
      \small 30/10/2014 & \small Adição do Datapath (Instruction Fetch) & \small santana22 e gabri14el \\ \hline
      \small 30/10/2014 & \small Alterações na subseção de acesso à memória com definições de número de bits & \small weversongomes \\ \hline
      \small 03/11/2014 & \small Correções na subseção "Leitura da Instrução" & \small weversongomes \\ \hline
      \small 05/11/2014 & \small Modificação dos datapaths internos & \small tarleswalker \\ \hline
      \small 05/11/2014 & \small Alteração nos opcodes & \small manuellemacedo \\ \hline
      \small 27/11/2014 & \small Refatoração da introdução e codificação & \small manuellemacedo \\ \hline
      \small 06/12/2014 & \small Atualização das informações do Instruction Fetch & \small santana22 \\ \hline
      \small 10/12/2014 & \small Atualização da tabela de microinstruções  & \small mirelarios \\ \hline
      \small 10/12/2014 & \small Modificando as informações contidas nos diagramas de classe de cada estágio  & \small  santana22 \\ \hline
      \small 11/12/2014 & \small Refatoração da introdução e codificação & \small manuellemacedo \\ \hline
      \small 11/12/2014 & \small Alteração no datapath do Intruction Fetch  & \small mirelarios \\ \hline
      \small 14/12/2014 & \small Atualização dos diagramas de classe & \small santana22 \\ \hline
      \small 15/12/2014 & \small Adição dos datapaths dos estágios 2 e 3 & \small mirelarios \\ \hline
	\end{longtable}
	\end{table}

% TOC instantiation
\tableofcontents

%%%%%%%%%%%%%%%%%%%%%%%%%%%%%%%%%%%%%%%%%%%%%%%%%%
%% Document main content
%%%%%%%%%%%%%%%%%%%%%%%%%%%%%%%%%%%%%%%%%%%%%%%%%%
\chapter{Introdução}
  
	  \section{Propósito do Documento}
  Este documento descreve a arquitetura do projeto MUSA. Como parte integrante deste documento estão os diagramas de classe, as definições de entrada e saída como também, as especificações dos circuitos internos de cada componente. Dessa forma, o principal objetivo deste documento é definir as especificações do projeto MUSA.

  \section{Stakeholders}
    \FloatBarrier
    \begin{table}[H] 
      \begin{center}
        \begin{tabular}[pos]{|m{6cm} | m{8cm}|} 
          \hline 
          \cellcolor[gray]{0.9}\textbf{Nome} & \cellcolor[gray]{0.9}\textbf{Papel/Responsabilidades} \\  
           \hline Manuelle 	& Gerência \\
           \hline Manuelle, Patrick, Lucas, Mirela, Vinícius, Gabriel, Weverson, Anderson e Tarles 	& Análise \\
           \hline Manuelle, Patrick, Lucas, Mirela, Vinícius, Gabriel, Weverson, Anderson e Tarles 	& Projeto \\ \hline
        \end{tabular}
      \end{center}
    \end{table} 

\section{Visão Geral do Documento}

O presente documento é apresentado como segue:

  \begin{itemize}
   \item \textbf{Capítulo 2 --} Este capítulo apresenta uma visão geral da arquitetura, com foco nas entradas e saídas do sistema e na arquitetura geral do mesmo;
   \item \textbf{Capítulo 3 --} Este capítulo descreve a arquitetura interna do IP a partir do detalhamento dos seus componentes, da definição das portas de entrada e saída e das especificações do caminho de dados (\textit{datapath}).
  \end{itemize}

  % inicio da tabela de acronimos e abreviacoes do documento
  \section{Acrônimos e Abreviações}
    \FloatBarrier
    \begin{table}[H]
      \begin{center}
        \begin{tabular}[pos]{|m{2cm} | m{12cm}|} 
          \hline
          \cellcolor[gray]{0.9}\textbf{Sigla} & \cellcolor[gray]{0.9}\textbf{Descrição} \\ \hline
          PC      &  Program Counter  \\ \hline
          OPCODE  &  Operation Code  \\ \hline
          RF  &  Register Flag  \\ \hline
          CST  &  Constant  \\ \hline
        \end{tabular}
      \end{center}
    \end{table}  
  % fim

\chapter{Visão Geral da Arquitetura}

%	\section{Restrições}
	\begin{itemize}
	\item \textbf{Restrições --} %fazer junto com os desenvolvedores
	\end{itemize}
	
	\section{Codificação das instruções}
	Instrução é uma palavra da linguagem de máquina, sua codificação é de fundamental importância para o processamento das operações.	 
  	\\Todas as instruções contém 32 bits. Exitem 4 formatos de instruções: R (R-type), I (I-type), Load/Store e Jump.
  	\\O formato R está relacionado as instruções lógicas e aritméticas.
	\begin{figure}[H]
    	\centering
    	\includegraphics{r-format}
    	\caption{Formato R}
		\label{r_format}
	\end{figure}
	Seus respectivos campos são:
	\begin{itemize}
	\item \textbf{OPCODE} - Código da operação básica da instrução.
	\item \textbf{RS} - Registrador do primeiro operando de origem.
	\item \textbf{RT} - Registrador do segundo operando de origem.
	\item \textbf{RD} - Registrador destino.
	\item \textbf{SHAMT} - \textit{Shift amount}; Quantidade de deslocamento.
	\item \textbf{FUNCT} - Função; Esse campo seleciona a variante específica da operação no campo opcode, e as vezes, é chamado de código de função.
\end{itemize}	  	
  	Um segundo tipo de formato de instrução é chamado de formato I, utilizado pelas instruções imediatas e de transferência de dados.
	\begin{figure}[H]
    	\centering
    	\includegraphics{i-format}
    	\caption{Formato I}
		\label{i_format}
  	\end{figure}
Seus respectivos campos são:
	\begin{itemize}
	\item \textbf{OPCODE} - Código da operação básica da instrução.
	\item \textbf{RS} - Registrador do operando de origem.
	\item \textbf{RT} - Registrador destino.
	\item \textbf{ADDRESS} - Endereço de memória ou constante numérica.
\end{itemize}	  	
  	
  	\begin{figure}[H]
    	\centering
    	\includegraphics{load-store}
    	\caption{Formato Load/Store}
		\label{loadstore}
  	\end{figure}
  	
   	\begin{figure}[H]
    	\centering
    	\includegraphics{jump}
    	\caption{Formato Jump}
		\label{jump}
  	\end{figure}
  	
 
	
	\section{Descrição dos Componentes}
  A unidade de processamento a ser desenvolvida é composta a partir dos seguintes componentes:

  \begin{itemize}
    \item \textbf{Serial Controller --} Controlador para comunicação com módulo de transmissão serial através do protocolo RS232.
    \item \textbf{Interface Control --} Interface de controle, responsável por fazer a leitura correta das informações da serial e transmiti-las para a unidade de processamento.
    \item \textbf{Processing Unit --} Unidade responsável pela realização das operações e armazenamento do resultado.
  \end{itemize}

 % inicio das descrições de arquitetura para cada componente do sistema
\chapter{Descrição da Arquitetura}

	\begin{center}
	\begin{tikzpicture}
		\umlclass[x=0,y=0]{Instruction Fetch}{
				+ clock : input bit \\
				+ pcInput : input bit[32] \\ 
				+ pcWrite : input bit \\
				+ pcOutput : output bit[32] \\
				+ instruction : output bit[32] \\
			}			
			{
			
			}
	\end{tikzpicture}
\end{center}
	\newpage
	
	\begin{center}
	\begin{tikzpicture}
	\umlclass[x=0,y=0]{Instruction Decode}{
	+ clock : input bit \\
	+ reset : input bit \\
	+ instruction : input bit[32] \\ 
	+ regDst : input bit[5] \\
	+ writeData : input bit \\
	+ writeRegister : input bit[5] \\
	+ word : input bit[16] \\
	+ branch : output bit \\
	+ memRead : output bit \\
	+ memToReg : output bit \\
	+ aluOp : output bit[2] \\
	+ memWrite : output bit \\
	+ aluSrc : output bit \\
	+ regWrite : output bit \\
	+ jump : output bit \\
	%+ readData1 : output bit[32] \\
	%+ readData2 : output bit [32] \\
	%+ outputWord : output bit [32] \\
	- registers : reg bit[32] \\}			
	{ % procedures
          - \underline{<<comb>> opcode\_decoder()} \\
          - <<comb>> search\_register() \\
          - <<comb>> set\_write\_register() \\
          - <<sequ>> sign\_extend() \\
          - <<sequ>> zero\_extend()
        }
	\end{tikzpicture}
\end{center}
	\newpage
	
	\section{Estágio de execução}
	\subsection{Diagrama de Classe}
  \begin{figure}[h]
    \begin{center}
	\begin{tikzpicture}
	\umlclass[x=0,y=0]{EX}{
	+ data_a : input bit[32] \\
	+ data_b : input bit[32] \\ 
	+ pc_in : input bit[32] \\ 
	+ orig_pc: input bit [2] \\
	+ op_alu : input bit[5] \\
	+ origin_aalu : input bit[1]\\
	+ origin_balu : input bit[2] \\
	+ instruction : output bit[32] \\
	+ offset_inst : input [32] \\}			
	{+<<comb>> calc_next_pc() \\
	+ <<comb>> cal_al_operation() \\}
	\end{tikzpicture}
\end{center}

  \end{figure}
		
		\subsection{Definições de entrada e saída}
		
	\begin{center}
		\begin{longtable}[pos]{| l | c | c | m{7cm} |} \hline
			\multicolumn{1}{|c|}{\cellcolor[gray]{0.9}\textbf{Nome}} & 
			\multicolumn{1}{c|}{\cellcolor[gray]{0.9}\textbf{Tamanho}} & 
			\multicolumn{1}{c|}{\cellcolor[gray]{0.9}\textbf{Direção}} &
			\multicolumn{1}{c|}{\cellcolor[gray]{0.9}\textbf{Descrição}} \\ \hline
			\endhead
			\hline
			\endlastfoot
			
			data\_a & 32 & Entrada & Dado do primeiro operando. \\ \hline
			data\_b & 32 & Entrada & Dado do segundo operando. \\ \hline
			pc\_in & 32 & Entrada & Valor do PC atual. \\ \hline
			orig\_pc & 2 & Define qual a origem do próximo pc, ou seja, se é branch, jump ou a próxima instrução \\ \hline
			
			
			
			
		\end{longtable}
	\end{center}
	
	\subsection{Datapath Interno}
	
	\begin{figure}[ht]
		\begin{center}
		\includegraphics[scale = 0.5]{./datapath/step3.png}
		\caption*{Datapath interno do estágio de execução (EX).}
		\end{center}
	\end{figure}
	
	\newpage
	
	\begin{center}
	\begin{tikzpicture}
	\umlclass[x=0,y=0]{Acess Memory}{
	+ clock : input bit \\
	+ reset : input bit \\ 
	+ address : input bit[32] \\ 
	+ memRead : input bit \\ 
	+ memWrite : input bit\\ 
	+ memToReg : input bit \\
	+ data\_in : input bit[32] \\ 
	+ writeBack : output bit[32]\\}
	{\underline{+<<comb>> accessMem()}\\
	+<<comb>> writeReg()}
	\end{tikzpicture}
\end{center}
	\newpage
	
	\begin{landscape}
	\section{Datapath Externo}
	\begin{figure}[htpb!]
    	\includegraphics{./datapath/datapath_final-1.eps}
  	\end{figure}
  	\end{landscape}

% Optional bibliography section
% To use bibliograpy, first provide the ipprocess.bib file on the root folder.
% \bibliographystyle{ieeetr}
% \bibliography{ipprocess}

\end{document}