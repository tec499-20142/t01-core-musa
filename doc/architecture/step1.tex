\section{Leitura da Instrução}
	\subsection{Diagrama de Classe}
		\begin{center}
			\begin{tikzpicture}
			\umlclass[x=0,y=0]{Instruction Fetch}
			{
				+ clock : input bit \\
				+ pcInput : input bit[TBD] \\ 
				+ pcWrite : input bit \\
				+ pcOutput : output bit[TBD] \\
				+ instruction : output bit[32] \\
			}			
			{
			
			}
			\end{tikzpicture}
		\end{center}
		
		\subsection{Definições de entrada e saída}
		
	\begin{center}
		\begin{longtable}[pos]{| l | c | c | m{7cm} |} \hline
			\multicolumn{1}{|c|}{\cellcolor[gray]{0.9}\textbf{Nome}} & 
			\multicolumn{1}{c|}{\cellcolor[gray]{0.9}\textbf{Tamanho}} & 
			\multicolumn{1}{c|}{\cellcolor[gray]{0.9}\textbf{Direção}} &
			\multicolumn{1}{c|}{\cellcolor[gray]{0.9}\textbf{Descrição}} \\ \hline
			\endhead
			\hline
			\endlastfoot
			
			pcInput & TBD & entrada & Valor do PC atual.\\ \hline
			pcWrite & 1 & entrada & Sinal proveniente da UC que habilita a modificação do valor de PC. \\ \hline
			pcOutput & TBD & saída & Valor do PC atual. \\ \hline
			instruction & 32 & saída & Instrução a ser executada. \\ \hline
			
		\end{longtable}
	\end{center}