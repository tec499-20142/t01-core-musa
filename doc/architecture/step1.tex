\section{Leitura da Instrução}
	\subsection{Diagrama de Classe}
  \begin{figure}[H]
    \begin{center}
	\begin{tikzpicture}
		\umlclass[x=0,y=0]{Instruction Fetch}{
				+ clock : input bit \\
				+ pcInput : input bit[32] \\ 
				+ pcWrite : input bit \\
				+ pcOutput : output bit[32] \\
				+ instruction : output bit[32] \\
			}			
			{
			
			}
	\end{tikzpicture}
\end{center}
  \end{figure}
		
		\subsection{Definições de entrada e saída}
		
	\begin{center}
		\begin{longtable}[pos]{| l | c | c | m{7cm} |} \hline
			\multicolumn{1}{|c|}{\cellcolor[gray]{0.9}\textbf{Nome}} & 
			\multicolumn{1}{c|}{\cellcolor[gray]{0.9}\textbf{Tamanho}} & 
			\multicolumn{1}{c|}{\cellcolor[gray]{0.9}\textbf{Direção}} &
			\multicolumn{1}{c|}{\cellcolor[gray]{0.9}\textbf{Descrição}} \\ \hline
			\endhead
			\hline
			\endlastfoot
			pcInput & 32 & entrada & Valor do PC atual\\ \hline
			pcWrite & 1 & entrada & Sinal que habilita a modificação do valor de PC \\ \hline
			pcOutput & 32 & saída & Valor do PC atual \\ \hline
			instruction & 32 & saída & Instrução a ser executada \\ \hline
			
		\end{longtable}
	\end{center}
	
	\subsection{Datapath Interno}
	\begin{figure}
		\begin{center}
		\includegraphics[scale=0.5]{./datapath/Graphic1.eps}
		\end{center}
	\end{figure}