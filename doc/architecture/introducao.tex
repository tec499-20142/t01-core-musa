\section{Propósito do Documento}
  Este documento descreve a arquitetura do projeto MUSA, incluindo especificações do circuitos internos de cada componente. Ele também apresenta diagramas de classe, definições de entrada e saída.O principal objetivo deste documento é definir as especificações do projeto MUSA e prover uma visão geral completa do mesmo.

  \section{Stakeholders}
    \FloatBarrier
    \begin{table}[H] 
      \begin{center}
        \begin{tabular}[pos]{|m{6cm} | m{8cm}|} 
          \hline 
          \cellcolor[gray]{0.9}\textbf{Nome} & \cellcolor[gray]{0.9}\textbf{Papel/Responsabilidades} \\  
           \hline Manuelle Macedo 	& Gerência \\
           \hline Patrick 	& Análise \\
           \hline Dilan Nery, Lucas Almeida, Mirela Rios, Cabele e Vinícius Santana	& Desenvolvimento \\
           \hline Antônio Gabriel e Weverson Gomes	& Testes       \\
           \hline Tarles Walker e Anderson Queiroz 	& Implementação \\ \hline
        \end{tabular}
      \end{center}
    \end{table} 

\section{Visão Geral do Documento}

O presente documento é apresentado como segue:

  \begin{itemize}
   \item \textbf{Capítulo 2 --} Este capítulo apresenta uma visão geral da arquitetura, com foco em entrada e saída do sistema e arquitetura geral do mesmo;
   \item \textbf{Capítulo 3 --} Este capítulo descreve a arquitetura interna do IP a partir do detalhamento dos seus componentes, definição de portas de entrada e saída e especificação de caminho de dados.
  \end{itemize}

  % inicio da tabela de acronimos e abreviacoes do documento
  \section{Acrônimos e Abreviações}
    \FloatBarrier
    \begin{table}[H]
      \begin{center}
        \begin{tabular}[pos]{|m{2cm} | m{12cm}|} 
          \hline
          \cellcolor[gray]{0.9}\textbf{Sigla} & \cellcolor[gray]{0.9}\textbf{Descrição} \\ \hline
          PC      &  Program Counter  \\ \hline
          OPCODE  &  Operation Code  \\ \hline
        \end{tabular}
      \end{center}
    \end{table}  
  % fim