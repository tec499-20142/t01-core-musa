  \section{Propósito do Documento}
  Este documento descreve a arquitetura do projeto MUSA. Como parte integrante deste documento estão os diagramas de classe, as definições de entrada e saída como também, as especificações dos circuitos internos de cada componente. Dessa forma, o principal objetivo deste documento é definir as especificações do projeto MUSA.

  \section{Stakeholders}
    \FloatBarrier
    \begin{table}[H] 
      \begin{center}
        \begin{tabular}[pos]{|m{6cm} | m{8cm}|} 
          \hline 
          \cellcolor[gray]{0.9}\textbf{Nome} & \cellcolor[gray]{0.9}\textbf{Papel/Responsabilidades} \\  
           \hline Manuelle 	& Gerência \\
           \hline Manuelle, Patrick, Lucas, Mirela, Vinícius, Gabriel, Weverson, Anderson e Tarles 	& Análise \\
           \hline Manuelle, Patrick, Lucas, Mirela, Vinícius, Gabriel, Weverson, Anderson e Tarles 	& Projeto \\ \hline
        \end{tabular}
      \end{center}
    \end{table} 

\section{Visão Geral do Documento}

O presente documento é apresentado como segue:

  \begin{itemize}
   \item \textbf{Capítulo 2 --} Este capítulo apresenta uma visão geral da arquitetura, com foco nas entradas e saídas do sistema e na arquitetura geral do mesmo;
   \item \textbf{Capítulo 3 --} Este capítulo descreve a arquitetura interna do IP a partir do detalhamento dos seus componentes, da definição das portas de entrada e saída e das especificações do caminho de dados (\textit{datapath}).
  \end{itemize}

  % inicio da tabela de acronimos e abreviacoes do documento
  \section{Acrônimos e Abreviações}
    \FloatBarrier
    \begin{table}[H]
      \begin{center}
        \begin{tabular}[pos]{|m{2cm} | m{12cm}|} 
          \hline
          \cellcolor[gray]{0.9}\textbf{Sigla} & \cellcolor[gray]{0.9}\textbf{Descrição} \\ \hline
          PC      &  Program Counter  \\ \hline
          OPCODE  &  Operation Code  \\ \hline
          RF  &  Register Flag  \\ \hline
          CST  &  Constant  \\ \hline
        \end{tabular}
      \end{center}
    \end{table}  
  % fim