\usecase{Instrução AND}

Este caso de uso especifica a ação da instrução AND (também chamada de conjunção lógica), que realiza a operação lógica entre dois valores.
 
\subsubsection*{Atores}
\textbf{IF, ID, EX, MEM/WB}

\preconditions 
\begin{itemize}
 \item O módulo \textbf{IF} precisa buscar uma instrução AND.
\end{itemize}

\postconditions
\begin{itemize}	
  \item Permitir execução de futuras instruções.
\end{itemize}

\subsubsection*{Fluxo Básico}
\begin{itemize}
\item O módulo \textbf{IF} busca a instrução AND a ser executada.
\item O módulo \textbf{ID} decodifica o tipo da instrução, emite sinais de controle para os demais módulos e disponibiliza os valores para o módulo \textbf{EX}.
\item O módulo \textbf{EX} realiza a operação lógica determinada dos valores e atualiza para a próxima instrução.
\item O módulo \textbf{MEM/WB} armazena o resultado da operação no módulo \textbf{ID}.
\end{itemize}

\subsubsection*{Fluxos Alternativos}
\begin{itemize}
\item Esta instrução não possui fluxos alternativos.
\end{itemize}

%\begin{figure}[htpb!]
%	\subsubsection*{Diagrama de Caso de Uso}
%		\flushleft
  \begin{tikzpicture} 
    \begin{umlsystem}[x=0, fill=red!10]{Buscar Instrução} 
      \umlusecase[x=-4,y=-3,name=buscar]{Buscar} 
      \umlusecase[x=0,y=-3, fill=green!20,width=1cm, name=enviar]{Enviar} 

      \umlactor[x=-8,y=-3]{Instruction Fetch} 
      \umlactor[x=4,y=-3]{Instruction Decode}   

      \umlassoc{Instruction Fetch}{buscar}
      \umlassoc{enviar}{Instruction Decode}
      \umlassoc{buscar}{enviar}
    \end{umlsystem} 
  \end{tikzpicture} 
%\end{figure}

\usecase{Instrução OR}

Este caso de uso especifica a ação da instrução OR (também chamada de disjunção lógica), que realiza a operação lógica entre dois valores.
 
\subsubsection*{Atores}
\textbf{IF, ID, EX, MEM/WB}

\preconditions 
\begin{itemize}
 \item O módulo \textbf{IF} precisa buscar uma instrução OR.
\end{itemize}

\postconditions
\begin{itemize}	
  \item Permitir execução de futuras instruções.
\end{itemize}

\subsubsection*{Fluxo Básico}
\begin{itemize}
\item O módulo \textbf{IF} busca a instrução OR a ser executada.
\item O módulo \textbf{ID} decodifica o tipo da instrução, emite sinais de controle para os demais módulos e disponibiliza os valores para o módulo \textbf{EX}.
\item O módulo \textbf{EX} realiza a operação lógica determinada dos valores e atualiza para a próxima instrução.
\item O módulo \textbf{MEM/WB} armazena o resultado da operação no módulo \textbf{ID}.
\end{itemize}

\subsubsection*{Fluxos Alternativos}
\begin{itemize}
\item Esta instrução não possui fluxos alternativos.
\end{itemize}

%\begin{figure}[htpb!]
%	\subsubsection*{Diagrama de Caso de Uso}
%		\flushleft
  \begin{tikzpicture} 
    \begin{umlsystem}[x=0, fill=red!10]{Buscar Instrução} 
      \umlusecase[x=-4,y=-3,name=buscar]{Buscar} 
      \umlusecase[x=0,y=-3, fill=green!20,width=1cm, name=enviar]{Enviar} 

      \umlactor[x=-8,y=-3]{Instruction Fetch} 
      \umlactor[x=4,y=-3]{Instruction Decode}   

      \umlassoc{Instruction Fetch}{buscar}
      \umlassoc{enviar}{Instruction Decode}
      \umlassoc{buscar}{enviar}
    \end{umlsystem} 
  \end{tikzpicture} 
%\end{figure}


\usecase{Instrução NOT}

Este caso de uso especifica a ação da instrução NOT (também chamada de inversora), pois realiza a negação (inversão) lógica de um valor.
 
\subsubsection*{Atores}
\textbf{IF, ID, EX, MEM/WB}

\preconditions 
\begin{itemize}
 \item O módulo \textbf{IF} precisa buscar uma instrução NOT.
\end{itemize}

\postconditions
\begin{itemize}	
  \item Permitir execução de futuras instruções.
\end{itemize}

\subsubsection*{Fluxo Básico}
\begin{itemize}
\item O módulo \textbf{IF} busca a instrução NOT a ser executada.
\item O módulo \textbf{ID} decodifica o tipo da instrução, emite sinais de controle para os demais módulos e disponibiliza o valor para o módulo \textbf{EX}.
\item O módulo \textbf{EX} realiza a operação lógica determinada de negação do valor e atualiza para a próxima instrução.
\item O módulo \textbf{MEM/WB} armazena o resultado da operação no módulo \textbf{ID}.
\end{itemize}

\subsubsection*{Fluxos Alternativos}
\begin{itemize}
\item Esta instrução não possui fluxos alternativos.
\end{itemize}

%\begin{figure}[htpb!]
%	\subsubsection*{Diagrama de Caso de Uso}
%		\flushleft
  \begin{tikzpicture} 
    \begin{umlsystem}[x=0, fill=red!10]{Buscar Instrução} 
      \umlusecase[x=-4,y=-3,name=buscar]{Buscar} 
      \umlusecase[x=0,y=-3, fill=green!20,width=1cm, name=enviar]{Enviar} 

      \umlactor[x=-8,y=-3]{Instruction Fetch} 
      \umlactor[x=4,y=-3]{Instruction Decode}   

      \umlassoc{Instruction Fetch}{buscar}
      \umlassoc{enviar}{Instruction Decode}
      \umlassoc{buscar}{enviar}
    \end{umlsystem} 
  \end{tikzpicture} 
%\end{figure}


\usecase{Instrução NOP}

Este caso de uso especifica a ação da instrução NOP, essa instrução não realiza nenhuma operação.
 
\subsubsection*{Atores}
\textbf{IF, ID, EX}

\preconditions 
\begin{itemize}
 \item O módulo \textbf{IF} precisa buscar uma instrução NOP.
\end{itemize}

\postconditions
\begin{itemize}	
  \item Permitir execução de futuras instruções.
\end{itemize}

\subsubsection*{Fluxo Básico}
\begin{itemize}
\item O módulo \textbf{IF} busca a instrução NOP a ser executada.
\item O módulo \textbf{ID} decodifica o tipo da instrução, emite sinais de controle para os demais módulos.
\item O módulo \textbf{EX} atualiza para a próxima instrução.
\end{itemize}

\subsubsection*{Fluxos Alternativos}
\begin{itemize}
\item Esta instrução não possui fluxos alternativos.
\end{itemize}

Basicamente as instruções lógicas e Aritméticas seguem um mesmo fluxo de execução.
\begin{figure}[htpb!]
	\subsubsection*{Diagrama de Caso de Uso}
		\flushleft
  \begin{tikzpicture} 
    \begin{umlsystem}[x=0, fill=red!10]{Buscar Instrução} 
      \umlusecase[x=-4,y=-3,name=buscar]{Buscar} 
      \umlusecase[x=0,y=-3, fill=green!20,width=1cm, name=enviar]{Enviar} 

      \umlactor[x=-8,y=-3]{Instruction Fetch} 
      \umlactor[x=4,y=-3]{Instruction Decode}   

      \umlassoc{Instruction Fetch}{buscar}
      \umlassoc{enviar}{Instruction Decode}
      \umlassoc{buscar}{enviar}
    \end{umlsystem} 
  \end{tikzpicture} 
\end{figure}
