  \usecase{Unidade de Controle}
 
    A \textbf{Unidade de Controle} é responsável por definir a partir de uma instrução decodificada o comportamento de outros módulos, 
    por exemplo: ULA. 
  
  \actors
    \begin{description}
     \actor{Controle}{Unidade que controla a execução das operações.}
    \end{description}
  
  \preconditions 
    \begin{itemize}
     \item Atender aos requisitos funcionais [FR27], [FR29-30];
     \item Identificação das unidades sequenciais e combinacionais;
    \end{itemize}

  \postconditions
    \begin{itemize}
      \item Analisar
    \end{itemize}
  
  % \ucdiagram{./pictures/uc_exemple.png}
  \subsubsection*{Diagrama de Caso de Uso}
    \begin{landscape}
  	  \flushleft
  \begin{tikzpicture}
  
      \umlactor[x=-2,y=0]{Controle} 
      \umlactor[x=7,y=-6.5]{ULA} 
       
    \begin{umlsystem}[x=0, fill=red!10]{Unidade de Controle} 
      \umlusecase[name=busca]{IF}
      \umlusecase[x=2,name=decod]{ID} 
      \umlusecase[x=5, name=sinais]{Transmite SC}    
    \end{umlsystem}
          
    \begin{umlsystem}[x=10,y=5, fill=red!10]{Banco de Registradores}
      \umlusecase[name=readregs]{Ler no Banco} 
      \umlusecase[x=4,name=writeregs]{Escrever no Banco}       
    \end{umlsystem}
               
    \begin{umlsystem}[x=10, fill=red!10]{Memória de Dados}     
      \umlusecase[name=readmemoria]{Ler na DM} 
      \umlusecase[x=4, name=writememoria]{Escrever na DM}
    \end{umlsystem}  
    
              
    \begin{umlsystem}[x=10,y=-6.5, fill=red!10]{ULA}            
      \umlusecase[name=operacao]{Efetuar OLA} 
      \umlusecase[x=4,y=2, width=1.5cm, name=processa]{Processar resultado}   
      \umlusecase[x=4,y=-2, fill=green!20,width=1cm, name=flags]{Definir flags}   
    \end{umlsystem} 
    
      \umlassoc{Controle}{busca}
      \umlassoc{busca}{decod}
      \umlassoc{decod}{sinais}
      \umlassoc{sinais}{readregs}
      \umlassoc{sinais}{writeregs}
      \umlassoc{sinais}{readmemoria}
      \umlassoc{sinais}{writememoria}
      \umlassoc{sinais}{operacao}
      \umlinclude{operacao}{processa}
      \umlassoc{operacao}{flags}
      \umlassoc{processa}{writeregs}
      \umlassoc{writeregs}{readmemoria}
      \umlassoc{readregs}{writememoria}
      \umlassoc{ULA}{operacao}
  \end{tikzpicture} 
 	\end{landscape}
 
  
  % descricao do fluxo principal de eventos
  \begin{mainflow}
    \item Decodificar o indicador operação;
    \item Realizar operação aritmética ou lógica;
    \item Armazenar o resultado em um registrador;
  \end{mainflow}
  
  % descricao do fluxo secundário (quando existir)
  \begin{secondaryflow} 
    \sfitem{Valor do resultado excede o suportado}
    \begin{enumerate}
      \item Habilitar sinal de overflow;
    \end{enumerate}
  \end{secondaryflow} 