  \usecase{Unidade de Controle}
 
    A \textbf{Unidade de Controle} é responsável por definir a partir de uma instrução decodificada o comportamento de outros módulos, 
    por exemplo: ULA. 
  
  \actors
    \begin{description}
     \actor{Controle}{Unidade que controla a execução das operações.}
    \end{description}
  
  \preconditions 
    \begin{itemize}
     \item Atender aos requisitos funcionais [FR20-24], [FR32-33];
     \item Identificação das unidades sequenciais e combinacionais;
     \item Busca da instrução.
    \end{itemize}

  \postconditions
    \begin{itemize}
      \item Ativar sinais de controle.
    \end{itemize}
  
  % \ucdiagram{./pictures/uc_exemple.png}
    \begin{landscape} 
    \begin{figure}
  \subsubsection*{Diagrama de Caso de Uso}
  	  \flushleft
  \begin{tikzpicture}
  
      \umlactor[x=-2,y=0]{Controle} 
      \umlactor[x=8,y=-4.5]{ULA} 
       
    \begin{umlsystem}[x=0, fill=red!10]{Unidade de Controle} 
      \umlusecase[name=busca]{IF}
      \umlusecase[x=1,name=decod]{ID} 
      \umlusecase[x=4, name=sinais]{Transmite SC}    
    \end{umlsystem}
          
    \begin{umlsystem}[x=4,y=4, fill=red!10]{Banco de Registradores}
      \umlusecase[name=readregs]{Ler no Banco} 
      \umlusecase[y=-2,name=writeregs]{Escrever no Banco}       
    \end{umlsystem}
               
    \begin{umlsystem}[x=4, fill=red!10]{Memória de Dados}     
      \umlusecase[name=readmemoria]{Ler na DM} 
      \umlusecase[y=-2, name=writememoria]{Escrever na DM}
    \end{umlsystem}  
    
              
    \begin{umlsystem}[x=4,y=-4, fill=red!10]{ULA}            
      \umlusecase[name=ula]{Efetuar OLA} 
      \umlusecase[x=4,y=2, width=1.5cm, name=processa]{Processar resultado}   
      \umlusecase[x=4,y=-2, fill=green!20,width=1cm, name=flags]{Definir flags}   
    \end{umlsystem} 
    
      \umlassoc{Controle}{busca}
      \umlassoc{busca}{decod}
      \umlassoc{decod}{sinais}
      \umlassoc{sinais}{readregs}
      \umlassoc{sinais}{writeregs}
      \umlassoc{sinais}{readmemoria}
      \umlassoc{sinais}{writememoria}
      \umlassoc{sinais}{ula}
      \umlinclude{ula}{processa}
      \umlassoc{ula}{flags}
      \umlassoc{processa}{writeregs}
      \umlassoc{writeregs}{readmemoria}
      \umlassoc{readregs}{writememoria}
      \umlassoc{ULA}{ula}
  \end{tikzpicture} 
  	\end{figure}
 	\end{landscape}
 
  
  % descricao do fluxo principal de eventos
  \begin{mainflow}
    \item Decodificar a instrução;
    \item Ativar os módulos que serão utilizado na instrução a medida que orem necessários.
  \end{mainflow}
  
  % descricao do fluxo secundário (quando existir)
%  \begin{secondaryflow} 
%    \sfitem{o}
%    \begin{enumerate}
%      \item ;
%    \end{enumerate}
%  \end{secondaryflow} 