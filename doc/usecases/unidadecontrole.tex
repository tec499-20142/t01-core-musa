\usecase{Unidade de Controle}
 
    A \textbf{Unidade de Controle} é responsável por definir, a partir de uma instrução decodificada, o comportamento de outros módulos.
  
  \actors
    \begin{description}
     \actor{UC}{Unidade que controla a execução das operações.}
    \end{description}
  
  \preconditions 
    \begin{itemize}
     \item Atender ao requisito funcional [FR23];
     \item Receber opcode da instrução.
    \end{itemize}

  \postconditions
    \begin{itemize}
      \item Ativar sinais de controle.
    \end{itemize}
  
\begin{landscape} 
    \begin{figure}
  \subsubsection*{Diagrama de Caso de Uso}
  	\input{./usecases/diagramUseCase.tikz}
  	\end{figure}
\end{landscape}
 
  
  % descricao do fluxo principal de eventos
  \begin{mainflow}
    \item Decodificar a instrução;
  \end{mainflow}
  
  % descricao do fluxo secundário (quando existir)
  \begin{secondaryflow} 
%    \sfitem{o}
%   \begin{enumerate}
     \item Ativar sinais de controle para instruções do tipo R.
     \item Ativar sinais de controle para instruções do tipo I.
     \item Ativar sinais de controle para instruções do tipo J.
%    \end{enumerate}
 \end{secondaryflow}