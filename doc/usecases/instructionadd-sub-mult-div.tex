\usecase{Instrução ADD}

Este caso de uso especifica a ação de execução de uma instrução de soma. O objetivo é realizar uma soma entre dois valores.
 
\subsubsection*{Atores}
\textbf{IF, ID, EX, MEM/WB}

\preconditions 
\begin{itemize}
 \item O módulo \textbf{IF} precisa buscar uma instrução ADD.
\end{itemize}

\postconditions
\begin{itemize}
  \item Permitir a execução da próxima instrução.
\end{itemize}

\subsubsection*{Fluxo Básico}
\begin{itemize}
\item O módulo \textbf{IF} busca a instrução ADD a ser executada.
\item O módulo \textbf{ID} decodifica o tipo da instrução, emite sinais de controle para os demais módulos e disponibiliza os valores para o módulo \textbf{EX}.
\item O módulo \textbf{EX} realiza a operação de soma dos dois valores e atualiza para o endereço da próxima instrução.
\item O módulo \textbf{MEM/WB} armazena o resultado no módulo \textbf{ID}.
\end{itemize}

\subsubsection*{Fluxos Alternativos}
\begin{itemize}
\item O resultado da operação pode não estar na faixa de valores suportados:
\subitem O módulo \textbf{EX} detecta e registra a ocorrência.
\end{itemize}

%\begin{figure}[htpb!]
%	\subsubsection*{Diagrama de Caso de Uso}
%		\flushleft
  \begin{tikzpicture} 
    \begin{umlsystem}[x=0, fill=red!10]{Instruções Aritméticas} 
      \umlusecase[x=-4,y=-3,name=buscar]{Buscar} 
      \umlusecase[x=0,y=-3, fill=green!20,width=1cm, name=enviar]{Enviar} 

      \umlactor[x=-8,y=-3]{IF} 
      \umlactor[x=4,y=-3]{ID}   

      \umlassoc{Instruction Fetch}{buscar}
      \umlassoc{enviar}{Instruction Decode}
      \umlassoc{buscar}{enviar}
    \end{umlsystem} 
  \end{tikzpicture} 
%\end{figure}

\usecase{Instrução SUB}

Este caso de uso especifica a ação da instrução de subtração. O objetivo é realizar a subtração entre dois valores.
 
\subsubsection*{Atores}
\textbf{IF, ID, EX, MEM/WB}

\preconditions 
\begin{itemize}
 \item O módulo \textbf{IF} precisa buscar uma instrução SUB.
\end{itemize}

\postconditions
\begin{itemize}
  \item Permitir a execução da próxima instrução.
\end{itemize}

\subsubsection*{Fluxo Básico}
\begin{itemize}
\item O módulo \textbf{IF} busca a instrução SUB a ser executada.
\item O módulo \textbf{ID} decodifica o tipo da instrução, emite sinais de controle para os demais módulos e disponibiliza os valores para o módulo \textbf{EX}.
\item O módulo \textbf{EX} realiza a operação de subtração dos dois valores e atualiza para o endereço da próxima instrução.
\item O módulo \textbf{MEM/WB} armazena o resultado da subtração no módulo \textbf{ID}.
\end{itemize}

\subsubsection*{Fluxos Alternativos}
\begin{itemize}
\item O resultado da operação pode não estar na faixa de valores suportados:
\subitem O módulo \textbf{EX} detecta e registra a ocorrência.
\end{itemize}

\usecase{Instrução MUL}

Este caso de uso especifica a ação da instrução de multiplicação. Seu objetivo é realizar a multiplicação entre dois valores.
 
\subsubsection*{Atores}
\textbf{IF, ID, EX, MEM/WB}

\preconditions 
\begin{itemize}
 \item O módulo \textbf{IF} precisa buscar uma instrução MUL.
\end{itemize}

\postconditions
\begin{itemize}
  \item Permitir a execução da próxima instrução.
\end{itemize}

\subsubsection*{Fluxo Básico}
\begin{itemize}
\item O módulo \textbf{IF} busca a instrução MUL a ser executada.
\item O módulo \textbf{ID} decodifica o tipo da instrução, emite sinais de controle para os demais módulos e disponibiliza os valores para o módulo \textbf{EX}.
\item O módulo \textbf{EX} realiza a operação de multiplicação dos dois valores e atualiza para o endereço da próxima instrução.
\item O módulo \textbf{MEM/WB} armazena o resultado da multiplicação no módulo \textbf{ID}.
\end{itemize}

\subsubsection*{Fluxos Alternativos}
\begin{itemize}
\item O resultado da operação pode não estar na faixa de valores suportados:
\subitem O módulo \textbf{EX} detecta e registra a ocorrência.
\end{itemize}

\usecase{Instrução DIV}

Este caso de uso especifica a ação da instrução de divisão. O objetivo é realizar a divisão entre dois valores.
 
\subsubsection*{Atores}
\textbf{IF, ID, EX, MEM/WB}

\preconditions 
\begin{itemize}
 \item O módulo \textbf{IF} precisa buscar uma instrução DIV.
\end{itemize}

\postconditions
\begin{itemize}	
  \item Permitir a execução da próxima instrução.
\end{itemize}

\subsubsection*{Fluxo Básico}
\begin{itemize}
\item O módulo \textbf{IF} busca a instrução DIV a ser executada.
\item O módulo \textbf{ID} decodifica o tipo da instrução, emite sinais de controle para os demais módulos e disponibiliza os valores para o módulo \textbf{EX}.
\item O módulo \textbf{EX} realiza a operação de divisão dos valores e atualiza para o endereço da próxima instrução.
\item O módulo \textbf{MEM/WB} armazena o resultado da divisão no módulo \textbf{ID}.
\end{itemize}

\subsubsection*{Fluxos Alternativos}
\begin{itemize}
\item O resultado da operação pode não estar na faixa de valores suportados:
\subitem O módulo \textbf{EX} detecta e registra a ocorrência.
\item O divisor disponibilizado pelo módulo \textbf{ID} para o módulo \textbf{EX} for 0 (zero):
\subitem O módulo \textbf{EX} detecta e registra a ocorrência.
\end{itemize}

\usecase{Instrução ADDI}

Este caso de uso especifica a ação de execução de uma instrução de soma imediata. O objetivo é realizar uma soma entre dois valores sendo um destes imediato.

\subsubsection*{Atores}
\textbf{IF, ID, EX, MEM/WB}

\preconditions 
\begin{itemize}
	\item O módulo \textbf{IF} precisa buscar uma instrução ADDI.
\end{itemize}

\postconditions
\begin{itemize}
	\item Permitir a execução da próxima instrução.
\end{itemize}

\subsubsection*{Fluxo Básico}
\begin{itemize}
	\item O módulo \textbf{IF} busca a instrução ADDI a ser executada.
	\item O módulo \textbf{ID} decodifica o tipo da instrução, emite sinais de controle para os demais módulos e disponibiliza os valores para o módulo \textbf{EX}.
	\item O módulo \textbf{EX} realiza a operação de soma dos dois valores e atualiza para o endereço da próxima instrução.
	\item O módulo \textbf{MEM/WB} armazena o resultado no módulo \textbf{ID}.
\end{itemize}

\subsubsection*{Fluxos Alternativos}
\begin{itemize}
	\item O resultado da operação pode não estar na faixa de valores suportados:
	\subitem O módulo \textbf{EX} detecta e registra a ocorrência.
\end{itemize}

%\begin{figure}[htpb!]
%	\subsubsection*{Diagrama de Caso de Uso}
%		\flushleft
  \begin{tikzpicture} 
    \begin{umlsystem}[x=0, fill=red!10]{Instruções Aritméticas} 
      \umlusecase[x=-4,y=-3,name=buscar]{Buscar} 
      \umlusecase[x=0,y=-3, fill=green!20,width=1cm, name=enviar]{Enviar} 

      \umlactor[x=-8,y=-3]{IF} 
      \umlactor[x=4,y=-3]{ID}   

      \umlassoc{Instruction Fetch}{buscar}
      \umlassoc{enviar}{Instruction Decode}
      \umlassoc{buscar}{enviar}
    \end{umlsystem} 
  \end{tikzpicture} 
%\end{figure}

\usecase{Instrução SUBI}

Este caso de uso especifica a ação de execução de uma instrução de subtração imediata. O objetivo é realizar uma subtração entre dois valores sendo um destes imediato.

\subsubsection*{Atores}
\textbf{IF, ID, EX, MEM/WB}

\preconditions 
\begin{itemize}
	\item O módulo \textbf{IF} precisa buscar uma instrução SUBI.
\end{itemize}

\postconditions
\begin{itemize}
	\item Permitir execução da próxima instrução.
\end{itemize}

\subsubsection*{Fluxo Básico}
\begin{itemize}
	\item O módulo \textbf{IF} busca a instrução SUBI a ser executada.
	\item O módulo \textbf{ID} decodifica o tipo da instrução, emite sinais de controle para os demais módulos e disponibiliza os valores para o módulo \textbf{EX}.
	\item O módulo \textbf{EX} realiza a operação de subtração dos dois valores e atualiza para o endereço da próxima instrução.
	\item O módulo \textbf{MEM/WB} armazena o resultado da subtração no módulo \textbf{ID}.
\end{itemize}

\subsubsection*{Fluxos Alternativos}
\begin{itemize}
	\item O resultado da operação pode não estar na faixa de valores suportados:
	\subitem O módulo \textbf{EX} detecta e registra a ocorrência.
\end{itemize}

\usecase{Instrução MULI}

Este caso de uso especifica a ação de execução de uma instrução de multiplicação imediata. O objetivo é realizar uma multiplicação entre dois valores sendo um destes imediato.

\subsubsection*{Atores}
\textbf{IF, ID, EX, MEM/WB}

\preconditions 
\begin{itemize}
	\item O módulo \textbf{IF} precisa buscar uma instrução MULI.
\end{itemize}

\postconditions
\begin{itemize}
	\item Permitir a execução da próxima instrução.
\end{itemize}

\subsubsection*{Fluxo Básico}
\begin{itemize}
	\item O módulo \textbf{IF} busca a instrução MULI a ser executada.
	\item O módulo \textbf{ID} decodifica o tipo da instrução, emite sinais de controle para os demais módulos e disponibiliza os valores para o módulo \textbf{EX}.
	\item O módulo \textbf{EX} realiza a operação de multiplicação dos dois valores e atualiza para o endereço da próxima instrução.
	\item O módulo \textbf{MEM/WB} armazena o resultado da multiplicação no módulo \textbf{ID}.
\end{itemize}

\subsubsection*{Fluxos Alternativos}
\begin{itemize}
	\item O resultado da operação pode não estar na faixa de valores suportados:
	\subitem O módulo \textbf{EX} detecta e registra a ocorrência.
\end{itemize}