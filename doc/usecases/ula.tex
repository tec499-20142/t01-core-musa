\usecase{Unidade Lógica Aritmética}
A \textbf{Unidade Lógica Aritmética} é responsável por realizar as operações aritméticas e lógicas, de acordo com o código da operação.
  
\actors
	\begin{description}
	\actor{ULA}{Unidade que realiza as operações lógicas e aritméticas.}
	\end{description}
  
\preconditions 
	\begin{itemize}
	\item Atender aos requisitos funcionais [FR1-19];
	\item Codificação das operações deve ser definida;
	\end{itemize}

\postconditions
    \begin{itemize}
     \item O módulo deve ser capaz de detectar overflow e underflow aritmético;
     \item O módulo deve ser capaz de ativar flags;
     \item O resultado deve estar armazenada no banco de registradores; %necessário?
     \end{itemize} 
  
%descricao do fluxo principal de eventos
\begin{mainflow}
\item Decodificar o indicador operação;
\item Realizar operação aritmética ou lógica;
\item Armazenar o resultado no banco de registradores;
\end{mainflow}
  
% descricao do fluxo secundário (quando existir)
\begin{secondaryflow} 
\sfitem{Valor do resultado excede o suportado}
\begin{enumerate}
\item Habilitar sinal de overflow ou underflow;
\end{enumerate}
\sfitem{Operação que ativa flags}
\begin{enumerate}
\item Habilitar um dos sinais de flags;
\end{enumerate}
\end{secondaryflow} 
