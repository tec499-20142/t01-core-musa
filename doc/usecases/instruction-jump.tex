\usecase{Instrução JR}

Este caso de uso especifica a ação da instrução de desvio incondicional JR. O objetivo é realizar um salto para o endereço armazenado em um registrador.
 
\subsubsection*{Atores}
\textbf{IF, ID, EX}

\preconditions 
\begin{itemize}
 \item O módulo \textbf{IF} precisa buscar uma instrução JR.
\end{itemize}

\postconditions
\begin{itemize}
  \item Permitir a execução da instrução endereçada pela conteúdo do registrador.
\end{itemize}

\subsubsection*{Fluxo Básico}
\begin{itemize}
\item O módulo \textbf{IF} busca a instrução JR a ser executada.
\item O módulo \textbf{ID} decodifica o tipo da instrução, emite sinais de controle para os demais módulos e seleciona o conteúdo do registrador.
\item O módulo \textbf{EX} atualiza para o endereço para a instrução do salto.
\end{itemize}

\usecase{Instrução JPC}

Este caso de uso especifica a ação da instrução de desvio incondicional JPC. O objetivo é realizar um salto para o endereço relativo ao endereço da instrução atual.
 
\subsubsection*{Atores}
\textbf{IF, ID, EX}

\preconditions 
\begin{itemize}
 \item O módulo \textbf{IF} precisa buscar uma instrução JPC.
\end{itemize}

\postconditions
\begin{itemize}
  \item Permitir a execução da instrução com o endereço relativo ao endereço da instrução atual.
\end{itemize}

\subsubsection*{Fluxo Básico}
\begin{itemize}
\item O módulo \textbf{IF} busca a instrução JPC a ser executada.
\item O módulo \textbf{ID} decodifica o tipo da instrução, emite sinais de controle para os demais módulos e seleciona o valor do desvio.
\item O módulo \textbf{EX} realiza a operação de soma do endereço da instrução atual com o valor selecionado e atualiza o endereço da próxima instrução.
\end{itemize}

\usecase{Instrução BRFL}

Este caso de uso especifica a ação da instrução de desvio condicional BRFL. O objetivo é realizar um salto para um endereço com base nas ocorrências registradas pelo módulo \textbf{EX}.
 
\subsubsection*{Atores}
\textbf{IF, ID, EX}

\preconditions 
\begin{itemize}
 \item O módulo \textbf{IF} precisa buscar uma instrução BRFL.
\end{itemize}

\postconditions
\begin{itemize}
  \item Permitir a execução da próxima instrução.
  \item O registrador de ocorrências deve ser limpo.
\end{itemize}

\subsubsection*{Fluxo Básico}
\begin{itemize}
\item O módulo \textbf{IF} busca a instrução BRFL a ser executada.
\item O módulo \textbf{ID} decodifica o tipo da instrução, emite sinais de controle para os demais módulos.
\item O módulo \textbf{EX} verifica qual foi a ocorrência. Com base nisso, modifica o endereço da próxima instrução.
\end{itemize}

\usecase{Instrução CALL}

Este caso de uso especifica a ação da instrução de desvio incondicional CALL. O objetivo é realizar um salto para uma sub-rotina.

\subsubsection*{Atores}
\textbf{IF, ID, EX}

\preconditions 
\begin{itemize}
	\item O módulo \textbf{IF} precisa buscar uma instrução CALL.
\end{itemize}

\postconditions
\begin{itemize}
	\item Permitir a execução da sub-rotina.
\end{itemize}

\subsubsection*{Fluxo Básico}
\begin{itemize}
	\item O módulo \textbf{IF} busca a instrução CALL a ser executada.
	\item O módulo \textbf{ID} decodifica o tipo da instrução, emite sinais de controle para os demais módulos.
	\item O módulo \textbf{EX} atualiza para o endereço da sub-rotina.
\end{itemize}

\usecase{Instrução RET}

Este caso de uso especifica a ação da instrução de desvio incondicional RET. O objetivo é realizar um salto de uma sub-rotina.

\subsubsection*{Atores}
\textbf{IF, ID, EX}

\preconditions 
\begin{itemize}
	\item O módulo \textbf{IF} precisa buscar uma instrução RET.
\end{itemize}

\postconditions
\begin{itemize}
	\item Permitir a execução da próxima instrução cujo endereço está depois da instrução de CALL.
\end{itemize}

\subsubsection*{Fluxo Básico}
\begin{itemize}
	\item O módulo \textbf{IF} busca a instrução RET a ser executada.
	\item O módulo \textbf{ID} decodifica o tipo da instrução, emite sinais de controle para os demais módulos.
	\item O módulo \textbf{EX} atualiza o endereço da próxima instrução para endereço depois da instrução de CALL.
\end{itemize}

\usecase{Instrução NOP}

Este caso de uso especifica a ação da instrução NOP. O objetivo dessa instrução é realizar nenhuma operação.

\subsubsection*{Atores}
\textbf{IF, ID, EX}

\preconditions 
\begin{itemize}
	\item O módulo \textbf{IF} precisa buscar uma instrução NOP.
\end{itemize}

\postconditions
\begin{itemize}	
	\item Permitir execução das próximas instruções.
\end{itemize}

\subsubsection*{Fluxo Básico}
\begin{itemize}
	\item O módulo \textbf{IF} busca a instrução NOP a ser executada.
	\item O módulo \textbf{ID} decodifica o tipo da instrução, emite sinais de controle para os demais módulos.
	\item O módulo \textbf{EX} atualiza para a próxima instrução.
\end{itemize}

\usecase{Instrução HALT}

Este caso de uso especifica a ação da instrução HALT. O objetivo é finalizar a execução de novas instruções.

\subsubsection*{Atores}
\textbf{IF, ID, EX}

\preconditions 
\begin{itemize}
	\item O módulo \textbf{IF} precisa buscar uma instrução HALT.
\end{itemize}

\postconditions
\begin{itemize}
	\item Nenhuma instrução deve ser executada.
\end{itemize}

\subsubsection*{Fluxo Básico}
\begin{itemize}
	\item O módulo \textbf{IF} busca a instrução HALT a ser executada.
	\item O módulo \textbf{ID} decodifica o tipo da instrução, emite sinais de controle para os demais módulos.
	\item O módulo \textbf{EX} atualiza o valor da próxima instrução para o endereço da instrução atual (\textit{loop} infinito).
\end{itemize}

De maneira análoga as instruções Lógicas e Aritméticas, as instruções de Desvio seguem um mesmo fluxo de execução:

\begin{figure}[H]
	\subsubsection*{Diagrama de Caso de Uso}
	\flushleft
  \begin{tikzpicture} 
    \begin{umlsystem}[x=0, fill=red!10]{Instruções de Jump} 
      \umlusecase[x=-4,y=-3,name=buscar]{Buscar} 
      \umlusecase[x=0,y=-3, fill=green!20,width=1cm, name=enviar]{Enviar} 

	  \umlusecase[x=0,y=-5,name=decodificar]{Decodificar} 
      \umlusecase[x=-4,y=-5, fill=green!20,width=1cm, name=emitir]{Emitir}

	  \umlusecase[x=-4,y=-8,name=operar]{Operar/Calcular} 
      \umlusecase[x=0,y=-9, fill=green!20,width=1cm, name=mod_end]{Modificar Endereço}

      \umlactor[x=-8,y=-3]{IF} 
      \umlactor[x=4,y=-4]{ID}   
      \umlactor[x=-8,y=-6]{EX}

      \umlassoc{IF}{buscar}
      \umlassoc{enviar}{ID}
      \umlassoc{buscar}{enviar}
      
      \umlassoc{ID}{decodificar}
      \umlassoc{emitir}{EX}
      \umlassoc{decodificar}{emitir}

      \umlassoc{EX}{operar}
       \umlassoc{operar}{mod_end}
       \umlassoc{mod_end}{IF}
    \end{umlsystem} 
  \end{tikzpicture} 
\end{figure}
