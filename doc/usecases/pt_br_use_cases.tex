%
% Portuguese-BR vertion
% 
\documentclass{article}

\usepackage{ipprocess}
% Use longtable if you want big tables to split over multiple pages.
% \usepackage{longtable}
\usepackage[utf8]{inputenc} 
\usepackage[brazil]{babel} % Uncomment for portuguese
\usepackage{tikz}
\usepackage{tikz-uml}

\sloppy

\graphicspath{{./pictures/}} % Pictures dir
\makeindex
\begin{document}

\DocumentTitle{Documento de Casos de Uso}
\Project{Processador de Propósito Geral Simples}
\Organization{Universidade Estadual de Feira de Santana}
\Version{Build 1.0a}

\capa
\newpage

%%%%%%%%%%%%%%%%%%%%%%%%%%%%%%%%%%%%%%%%%%%%%%%%%%
%% Revision History
%%%%%%%%%%%%%%%%%%%%%%%%%%%%%%%%%%%%%%%%%%%%%%%%%%
\section*{\center Histórico de Revisões}
  \vspace*{1cm}
  \begin{table}[ht]
    \centering
    \begin{tabular}[pos]{|m{2cm} | m{7.2cm} | m{3.8cm}|} 
      \hline
      \cellcolor[gray]{0.9}
      \textbf{Date} & \cellcolor[gray]{0.9}\textbf{Descrição} & \cellcolor[gray]{0.9}\textbf{Autor(s)}\\ \hline
      
      \small 03/10/2014 & \small Document conception & \small manuellemacedo \\ \hline      
	
      \small 07/10/2014 & \small DAdição do caso de uso da unidade lógica e aritmética & \small manuellemacedo \\ \hline
    \end{tabular}
  \end{table}

\newpage

% TOC instantiation
\tableofcontents
\newpage

%%%%%%%%%%%%%%%%%%%%%%%%%%%%%%%%%%%%%%%%%%%%%%%%%%
%% Document main content
%%%%%%%%%%%%%%%%%%%%%%%%%%%%%%%%%%%%%%%%%%%%%%%%%%
\section{Introdução}

 A \textbf{Fazemos Qualquer Negócio Inc.} foi contratada para o desenvolvimento do \textit{IP-Core} \textbf{MUSA}, que é um micro processador de
 propósito geral que será utilizado em escolas de países africanos com o intuito de impulsionar o desenvolvimento deste continente. As seções
 subsequentes apresentam os requisitos funcionais e não funcionais como parte da fase de \textbf{Concepção} do IP-process.
 Os requisitos listados foram definidos a partir das necessidades do cliente para produção do \textit{MUSA}.

\section{Atores do Sistema}

  \begin{figure}[H]
    \centering
    \begin{tikzpicture} 
      \umlactor{Controle IF} 
      \umlactor[x=2]{Controle}
      \umlactor[x=4]{LED de overflow} 
      \umlactor[x=6.5]{LED de dados}  
    \end{tikzpicture} 
  \end{figure}

  \begin{description}
    \actor{Controle IF}{Entidade responsável por controlar os pacotes de entrada de dados.}    
    \actor{Controle}{Unidade que controla a execução das operações.}
    \actor{LED de overflow}{Interface de saída que aciona o LED de overflow.}
    \actor{LEDs de dados:}{Interface de saída que aciona os sinais de dados.}
  \end{description}

\section{Casos de Usos}

	  \usecase{Unidade de Controle}
 
    A \textbf{Unidade de Controle} é responsável por definir a partir de uma instrução decodificada o comportamento de outros módulos, 
    por exemplo: ULA. 
  
  \actors
    \begin{description}
     \actor{Controle}{Unidade que controla a execução das operações.}
    \end{description}
  
  \preconditions 
    \begin{itemize}
     \item Atender aos requisitos funcionais [FR27], [FR29-30];
     \item Identificação das unidades sequenciais e combinacionais;
    \end{itemize}

  \postconditions
    \begin{itemize}
      \item Analisar
    \end{itemize}
  
  % \ucdiagram{./pictures/uc_exemple.png}
  \subsubsection*{Diagrama de Caso de Uso}
    \begin{landscape}
  	  \flushleft
  \begin{tikzpicture}
  
      \umlactor[x=-2,y=0]{Controle} 
      \umlactor[x=7,y=-6.5]{ULA} 
       
    \begin{umlsystem}[x=0, fill=red!10]{Unidade de Controle} 
      \umlusecase[name=busca]{IF}
      \umlusecase[x=2,name=decod]{ID} 
      \umlusecase[x=5, name=sinais]{Transmite SC}    
    \end{umlsystem}
          
    \begin{umlsystem}[x=10,y=5, fill=red!10]{Banco de Registradores}
      \umlusecase[name=readregs]{Ler no Banco} 
      \umlusecase[x=4,name=writeregs]{Escrever no Banco}       
    \end{umlsystem}
               
    \begin{umlsystem}[x=10, fill=red!10]{Memória de Dados}     
      \umlusecase[name=readmemoria]{Ler na DM} 
      \umlusecase[x=4, name=writememoria]{Escrever na DM}
    \end{umlsystem}  
    
              
    \begin{umlsystem}[x=10,y=-6.5, fill=red!10]{ULA}            
      \umlusecase[name=operacao]{Efetuar OLA} 
      \umlusecase[x=4,y=2, width=1.5cm, name=processa]{Processar resultado}   
      \umlusecase[x=4,y=-2, fill=green!20,width=1cm, name=flags]{Definir flags}   
    \end{umlsystem} 
    
      \umlassoc{Controle}{busca}
      \umlassoc{busca}{decod}
      \umlassoc{decod}{sinais}
      \umlassoc{sinais}{readregs}
      \umlassoc{sinais}{writeregs}
      \umlassoc{sinais}{readmemoria}
      \umlassoc{sinais}{writememoria}
      \umlassoc{sinais}{operacao}
      \umlinclude{operacao}{processa}
      \umlassoc{operacao}{flags}
      \umlassoc{processa}{writeregs}
      \umlassoc{writeregs}{readmemoria}
      \umlassoc{readregs}{writememoria}
      \umlassoc{ULA}{operacao}
  \end{tikzpicture} 
 	\end{landscape}
 
  
  % descricao do fluxo principal de eventos
  \begin{mainflow}
    \item Decodificar o indicador operação;
    \item Realizar operação aritmética ou lógica;
    \item Armazenar o resultado em um registrador;
  \end{mainflow}
  
  % descricao do fluxo secundário (quando existir)
  \begin{secondaryflow} 
    \sfitem{Valor do resultado excede o suportado}
    \begin{enumerate}
      \item Habilitar sinal de overflow;
    \end{enumerate}
  \end{secondaryflow} 

	\usecase{Unidade Lógica Aritmética}
A \textbf{Unidade Lógica Aritmética} é responsável por realizar as operações aritméticas e lógicas, de acordo com o código da operação.
  
\actors
	\begin{description}
	\actor{ULA}{Unidade que realiza as operações lógicas e aritméticas.}
	\end{description}
  
\preconditions 
	\begin{itemize}
	\item Atender aos requisitos funcionais [FR1-19];
	\item Codificação das operações deve ser definida;
	\end{itemize}

\postconditions
    \begin{itemize}
     \item O módulo deve ser capaz de detectar overflow e underflow aritmético;
     \item O módulo deve ser capaz de ativar flags;
     \item O resultado deve estar armazenada no banco de registradores; %necessário?
     \end{itemize} 
  
%descricao do fluxo principal de eventos
\begin{mainflow}
\item Decodificar o indicador operação;
\item Realizar operação aritmética ou lógica;
\item Armazenar o resultado no banco de registradores;
\end{mainflow}
  
% descricao do fluxo secundário (quando existir)
\begin{secondaryflow} 
\sfitem{Valor do resultado excede o suportado}
\begin{enumerate}
\item Habilitar sinal de overflow ou underflow;
\end{enumerate}
\sfitem{Operação que ativa flags}
\begin{enumerate}
\item Habilitar um dos sinais de flags;
\end{enumerate}
\end{secondaryflow}
	
	%\usecase{Memória de Dados}
A \textbf{Mémoria de Dados} é responsável por armazenar os dados do programa.
  
\actors
	\begin{description}
	\actor{UC}{Unidade que controla a execução das operações.}
	\end{description}
  
\preconditions 
	\begin{itemize}
	\item Atender aos requisitos funcionais [FR25-26];
	\item O(s) dado(s) a serem armazenados devem estar no banco de registradores;
	\end{itemize}

\postconditions
    \begin{itemize}
     \item A memória não deve permitir sobrescrita dados quando não-autorizado;
     \end{itemize}
  
%descricao do fluxo principal de eventos
\begin{mainflow}
\item Recebecor o endereço de onde o dado será armazenado/lido;
\item Realizar a leitura/escrita do dado;
\end{mainflow}
  
% descricao do fluxo secundário (quando existir)
%\begin{} 
%\sfitem{}
%\begin{enumerate}
%\item ;
%\end{enumerate}
%\end{secondaryflow}
	
	%\input{}

  
% Optional bibliography section
% To use bibliograpy, first provide the ipprocess.bib file on the root folder.
% \bibliographystyle{ieeetr}
% \bibliography{ipprocess}

\end{document}
