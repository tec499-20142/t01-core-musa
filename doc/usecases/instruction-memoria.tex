\usecase{Instrução LW}

Este caso de uso especifica a ação da instrução LW. O objetivo é carregar um dado da memória para um registrador.
 
\subsubsection*{Atores}
\textbf{IF, ID, EX, MEM/WB}

\preconditions 
\begin{itemize}
 \item O módulo \textbf{IF} precisa buscar uma instrução LW.
\end{itemize}

\postconditions
\begin{itemize}
  \item Permitir a execução da próxima instrução.
\end{itemize}

\subsubsection*{Fluxo Básico}
\begin{itemize}
\item O módulo \textbf{IF} busca a instrução LW a ser executada.
\item O módulo \textbf{ID} decodifica o tipo da instrução, emite sinais de controle para os demais módulos e seleciona o conteúdo do registrador.
\item O módulo \textbf{EX} realiza o cálculo do endereçamento do dado na memória e atualiza para o endereço da próxima instrução.
\item O módulo \textbf{MEM/WB} armazena o dado buscado no módulo \textbf{ID}.
\end{itemize}

\usecase{Instrução SW}

Este caso de uso especifica a ação da instrução SW. O objetivo é armazenar um dado de um registrador na memória.

\subsubsection*{Atores}
\textbf{IF, ID, EX, MEM/WB}

\preconditions 
\begin{itemize}
	\item O módulo \textbf{IF} precisa buscar uma instrução SW.
\end{itemize}

\postconditions
\begin{itemize}
	\item Permitir a execução da próxima instrução.
\end{itemize}

\subsubsection*{Fluxo Básico}
\begin{itemize}
	\item O módulo \textbf{IF} busca a instrução SW a ser executada.
	\item O módulo \textbf{ID} decodifica o tipo da instrução, emite sinais de controle para os demais módulos e seleciona o conteúdo do registrador.
	\item O módulo \textbf{EX} realiza o cálculo do endereçamento do dado na memória e atualiza para o endereço da próxima instrução.
	
	\item O módulo \textbf{MEM/WB} armazena o dado buscado na memória no módulo \textbf{MEM/WB}.
\end{itemize}

\begin{figure}[H]
	\subsubsection*{Diagrama de Caso de Uso}
		\input{./usecases/instructionmemory.tikz}
\end{figure}
