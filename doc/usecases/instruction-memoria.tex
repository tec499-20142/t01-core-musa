\usecase{Instrução LW}

Este caso de uso especifica a ação da instrução LW. O objetivo é carregar um dado da memória para um registrador.
 
\subsubsection*{Atores}
\textbf{IF, ID, EX, MEM/WB}

\preconditions 
\begin{itemize}
 \item O módulo \textbf{IF} precisa buscar uma instrução LW.
\end{itemize}

\postconditions
\begin{itemize}
  \item Permitir a execução da próxima instrução.
\end{itemize}

\subsubsection*{Fluxo Básico}
\begin{itemize}
\item O módulo \textbf{IF} busca a instrução LW a ser executada.
\item O módulo \textbf{ID} decodifica o tipo da instrução, emite sinais de controle para os demais módulos e seleciona o conteúdo do registrador.
\item O módulo \textbf{EX} realiza o cálculo do endereçamento do dado na memória e atualiza para o endereço da próxima instrução.
\item O módulo \textbf{MEM/WB} armazena o dado buscado no módulo \textbf{ID}.
\end{itemize}

\usecase{Instrução SW}

Este caso de uso especifica a ação da instrução SW. O objetivo é armazenar um dado de um registrador na memória.

\subsubsection*{Atores}
\textbf{IF, ID, EX, MEM/WB}

\preconditions 
\begin{itemize}
	\item O módulo \textbf{IF} precisa buscar uma instrução SW.
\end{itemize}

\postconditions
\begin{itemize}
	\item Permitir a execução da próxima instrução.
\end{itemize}

\subsubsection*{Fluxo Básico}
\begin{itemize}
	\item O módulo \textbf{IF} busca a instrução SW a ser executada.
	\item O módulo \textbf{ID} decodifica o tipo da instrução, emite sinais de controle para os demais módulos e seleciona o conteúdo do registrador.
	\item O módulo \textbf{EX} realiza o cálculo do endereçamento do dado na memória e atualiza para o endereço da próxima instrução.
	
	\item O módulo \textbf{MEM/WB} armazena o dado buscado na memória no módulo \textbf{MEM/WB}.
\end{itemize}

\begin{figure}[H]
	\subsubsection*{Diagrama de Caso de Uso}
		\flushleft
  \begin{tikzpicture} 
    \begin{umlsystem}[x=0, fill=red!10]{Instruções de Acesso a Memória} 
      \umlusecase[x=-4,y=-3,name=buscar]{Buscar} 
      \umlusecase[x=0,y=-3, fill=green!20,width=1cm, name=enviar]{Enviar} 

	  \umlusecase[x=0,y=-5,name=decodificar]{Decodificar} 
      \umlusecase[x=-4,y=-5, fill=green!20,width=1cm, name=emitir]{Emitir}

	  \umlusecase[x=-4,y=-7,name=operar]{Operar/Calcular} 
      \umlusecase[x=0,y=-9, fill=green!20,width=1cm, name=emitir_resultado]{Emitir Resultado}
	  \umlusecase[x=0,y=-7, fill=green!20,width=1.5cm, name=armazenar]{Armazenar}

      \umlactor[x=-8,y=-3]{IF} 
      \umlactor[x=4,y=-4]{ID}   
      \umlactor[x=-8,y=-6]{EX}
      \umlactor[x=4,y=-7]{MEM/WB}   

      \umlassoc{IF}{buscar}
      \umlassoc{enviar}{ID}
      \umlassoc{buscar}{enviar}
      
      \umlassoc{ID}{decodificar}
      \umlassoc{emitir}{EX}
      \umlassoc{decodificar}{emitir}

      \umlassoc{EX}{operar}
      \umlassoc{emitir_resultado}{MEM/WB}
      \umlassoc{operar}{emitir_resultado}
	  \umlassoc{armazenar}{MEM/WB}
    \end{umlsystem} 

  \end{tikzpicture} 
\end{figure}
